\RequirePackage[hyphens]{url}
\RequirePackage{latexml}
\iflatexml
\documentclass[12pt,a4paper,twoside]{book}
\else
\documentclass[paper=a4,
					fontsize=12pt,
				headsepline,
				cleardoublepage=plain,
				numbers=noenddot,
				bibliography=totoc]{scrbook}
\fi

\usepackage[english]{babel}
\usepackage[utf8]{inputenc}     						
\usepackage[T1]{fontenc}								% T1-fonts

%\usepackage{mathptmx}									% Times/Mathe \rmdefault
%\usepackage[scaled=.90]{helvet}						% scaled Helvetica \sfdefault
\usepackage{helvet} 										% Helvetica \sfdefault
\usepackage{courier}       							% Courier \ttdefault


% Additional packages 
\usepackage{natbib}	          
\usepackage[intlimits]{amsmath}	
\usepackage{amsthm,amsfonts}		
\usepackage{graphicx}				
\usepackage{color}     				
\usepackage[font=small, format=plain, labelfont=bf]{caption}		
\usepackage[tight]{subfigure}		% support for subfigures (should be replaced
											% by the newer "subfig" package
\usepackage{units}					% support for physical units


% the hyperref package for hyperlinks should be included as last package

\usepackage[pdftex,
				breaklinks=true,  	% some links do not work, if this option is set
				colorlinks=true,  	% before printout, set to false!!!
				linkcolor=blue,
				citecolor=blue,
				urlcolor=blue,
				pagebackref=true]{hyperref}

% use normal fontstyle for urls; if not set, a tt-font is used
\urlstyle{same}

\pagestyle{headings}

% distance of line from head:
\headsep4mm

\iflatexml
%% nothing
\else
% recompute type area
\typearea[10mm]{current}		
\fi

% no indentation at the beginning of a new paragraph
%\setlength{\parindent}{0pt}

% new commands for math mode

% vector and absolute value
\newcommand{\abs}[1]{\left\lvert#1\right\rvert}
\renewcommand{\vec}[1]{\boldsymbol{#1}}

% vector operators
\DeclareMathOperator{\grad}{grad}
\DeclareMathOperator{\Div}{div}    %\div is used by amsmath
\DeclareMathOperator{\rot}{rot}

% brackets
\newcommand{\lra}[1]{ \left( #1 \right) }
\newcommand{\lrb}[1]{ \left[ #1 \right] }
\newcommand{\lrc}[1]{ \left\{ #1 \right\} }

% fractions
\newcommand{\td}[1]{\frac{d}{d #1}}
\newcommand{\ttd}[2]{\frac{d #2}{d #1}}
\newcommand{\pd}[1]{\frac{\partial}{\partial #1}}
\newcommand{\ppd}[2]{\frac{\partial #2}{\partial #1}}
\newcommand{\pdd}[1]{\frac{\partial^2}{\partial #1^2}}
\newcommand{\fa}{\frac{1}{a}}
\newcommand{\fh}{\frac{\dot{a}}{a}}
\newcommand{\ft}{\frac{1}{\sqrt{2\pi}}}
\newcommand{\fft}{\frac{1}{2\pi}}
\newcommand{\ffft}{\frac{1}{(2\pi)^{3/2}}}

% integrals
\newcommand{\iinf}{\int_{-\infty}^{\infty}}
\newcommand{\iiinf}{\iint_{-\infty}^{\infty}}
\newcommand{\iiiinf}{\iiint^{\infty}_{-\infty}}

% filtered quantities
%\newcommand{\fil}[1]{{<} #1 {>}}
\newcommand{\fil}[1]{\langle #1 \rangle}
\newcommand{\ffil}[1]{{\ll} #1 {\gg}}
\newcommand{\cfil}[1]{{\prec}#1{\succ}}
\newcommand{\chat}[1]{\accentset{\curlywedge}{#1}}
\newcommand{\ol}[1]{\overline{#1}}

\title{Vector calculus}
\author{Andreas Maier}
\date{\today}
\begin{document}
\maketitle
\tableofcontents

\section{Vector calculus}

\subsection{Longitudinal and transversal projection of vectors}

%\begin{figure}[htp]
%\centering
%\resizebox{0.4\textwidth}{!}{
%\input{projection.pstex_t}}
%\caption{Projection of a vector $\vec{a}$ on a vector $\vec{b}$.}
%\label{fig:projection}
%\end{figure}

A vector $\vec{a}$ can be split into two parts: the longitudinal part $\vec{a}_{\parallel}$, 
which is parallel to another vector $\vec{b}$ and the transversal part $\vec{a}_{\perp}$, 
which is perpendicular to $\vec{b}$. The length of the longitudinal part $a_{\parallel}$ 
and the transversal part $a_{\perp}$ can be computed from geometry (see figure \ref{fig:projection})
\begin{align}
\frac{a_{\parallel}}{a}=\cos \alpha & \Rightarrow
a_{\parallel}= a \cos\alpha = \frac{a b \cos \alpha}{b} = \frac{\vec{a}\cdot\vec{b}}{b},\\
\frac{a_{\perp}}{a}=\sin \alpha & \Rightarrow 
a_{\perp}= a \sin\alpha = \frac{a b \sin \alpha}{b} = \frac{\abs{\vec{a}\times\vec{b}}}{b}. \label{eq:protrans}
\end{align}
But from the Pythagorean theorem we can get another expression for the length of the transversal
part
\begin{align}
a^2=a_{\parallel}^2+a_{\perp}^2 & \Rightarrow
a_{\perp}^2 = a^2-a_{\parallel}^2= a^2-\frac{(\vec{a}\cdot\vec{b})^2}{b^2}.\label{eq:protrans2}
\end{align}
Substituting equation \eqref{eq:protrans} in equation \eqref{eq:protrans2} we get
\begin{align*}
\frac{(\abs{\vec{a}\times\vec{b}})^2}{b^2}=a^2-\frac{(\vec{a}\cdot\vec{b})^2}{b^2},
\end{align*}
which leads us to the following expression for the square of the norm of the cross product
\begin{align}
\abs{\vec{a}\times\vec{b}}^2 = (a b)^2 - (\vec{a}\cdot\vec{b})^2.
\end{align}
This is again Lagrange's identity (see \eqref{eq:lagrident}). 

\subsection{Vector identities}
In this chapter we show the derivation of some vector quantities in cartesian
tensor notation.
\subsubsection{$(\vec{u}\cdot\nabla) \vec{v}$} \label{vecid01}
For some arbitraty vectors $u_i, v_i$ we can write
\begin{align*}
u_j \pd{r_j} v_i &= \overbrace{u_j \pd{r_i} v_j - u_j \pd{r_i} v_j}^0 
+ u_j\pd{r_j} v_i \\
&= u_j \pd{r_i} v_j - \delta_{ik}\delta_{jl} u_j \pd{r_k} v_l 
+\delta_{il}\delta_{jk} u_j \pd{r_k} v_l \\
&= u_j \pd{r_i} v_j 
- (\delta_{ik}\delta_{jl}-\delta_{il}\delta_{jk})u_j \pd{r_k} v_l \\
&= u_j \pd{r_i} v_j - \epsilon_{mij} \epsilon_{mkl} u_j \pd{r_k} v_l \\
&= u_j \pd{r_i} v_j - \epsilon_{ijm} u_j \epsilon_{mkl} \pd{r_k} v_l.
\end{align*}
In vector notation this can be expressed like
\begin{align}
(\vec{u}\cdot\nabla) \vec{v} = 
\vec{u}\cdot (\nabla \vec{v})-\vec{u} \times (\nabla \times \vec{v})
\label{eq:vecid01}
\end{align}
\subsubsection{$(\vec{v}\cdot\nabla) \vec{v}$}
Inserting $u_j=v_j$ into equation \eqref{eq:vecid01} yields
\begin{align*}
v_j \pd{r_j} v_i = v_j \pd{r_i} v_j 
- \epsilon_{ijm} v_j \epsilon_{mkl} \pd{r_k} v_l
\end{align*}
For $v_j \pd{r_i} v_j$ we can write
\begin{align*}
v_j \pd{r_i} v_j &= \pd{r_i} (v_j v_j) - v_j \pd{r_i} v_j
\end{align*}
and therefore
\begin{align*}
v_j \pd{r_i} v_j = \pd{r_i} \lra{\frac{1}{2} v_j v_j}.
\end{align*}
Using this we get for
\begin{align*}
v_j \pd{r_j} v_i = \pd{r_i} \lra{\frac{1}{2} v_j v_j}
- \epsilon_{ijm} v_j \epsilon_{mkl} \pd{r_k} v_l
\end{align*}
or in vector notation
\begin{align}
(\vec{v}\cdot\nabla) \vec{v} = 
\frac{1}{2} \nabla \vec{v}^2-\vec{v} \times (\nabla \times \vec{v})
\label{eq:vecid02}
\end{align}
\subsubsection{$\nabla \times \lra{\vec{u} \times \vec{v}}$}
The $i$-th component of the rotation of a cross product of two vectors is
\begin{align*}
\lrb{\nabla \times \lra{\vec{u} \times \vec{v}}}_i 
&= \epsilon_{ijk} \pd{r_j} \epsilon_{klm} u_l v_m \\
&= \epsilon_{kij} \epsilon_{klm} \pd{r_j} (u_l v_m) \\
&= (\delta_{il}\delta_{jm}-\delta_{im}\delta_{jl}) \pd{r_j} (u_l v_m) \\
&= \pd{r_j} (u_i v_j) - \pd{r_j} (u_j v_i) \\
&= u_i \pd{r_j} v_j + v_j \pd{r_j} u_i - u_j \pd{r_j} v_i - v_i \pd{r_j} u_j
\end{align*}
It can be written in vector notation like
\begin{align}
\nabla \times \lra{\vec{u} \times \vec{v}} 
= \vec{u} (\nabla \cdot \vec{v}) - \vec{v} (\nabla \cdot \vec{u})
+ (\vec{v} \cdot \nabla) \vec{u} - (\vec{u} \cdot \nabla) \vec{v}
\end{align}

\appendix
%\input{appendix/appendix}

% try plainnat, abbrvnat, unsrtnat, apalike, aa, astroads

\bibliographystyle{apalike}
\bibliography{promotion}

\end{document}
