\RequirePackage[hyphens]{url}
\RequirePackage{latexml}
\iflatexml
\documentclass[12pt,a4paper,twoside]{book}
\else
\documentclass[paper=a4,
					fontsize=12pt,
				headsepline,
				cleardoublepage=plain,
				numbers=noenddot,
				bibliography=totoc]{scrbook}
\fi

\usepackage[english]{babel}
\usepackage[utf8]{inputenc}     						
\usepackage[T1]{fontenc}								% T1-fonts

%\usepackage{mathptmx}									% Times/Mathe \rmdefault
%\usepackage[scaled=.90]{helvet}						% scaled Helvetica \sfdefault
\usepackage{helvet} 										% Helvetica \sfdefault
\usepackage{courier}       							% Courier \ttdefault


% Additional packages 
\usepackage{natbib}	          
\usepackage[intlimits]{amsmath}	
\usepackage{amsthm,amsfonts}		
\usepackage{graphicx}				
\usepackage{color}     				
\usepackage[font=small, format=plain, labelfont=bf]{caption}		
\usepackage[tight]{subfigure}		% support for subfigures (should be replaced
											% by the newer "subfig" package
\usepackage{units}					% support for physical units


% the hyperref package for hyperlinks should be included as last package

\usepackage[pdftex,
				breaklinks=true,  	% some links do not work, if this option is set
				colorlinks=true,  	% before printout, set to false!!!
				linkcolor=blue,
				citecolor=blue,
				urlcolor=blue,
				pagebackref=true]{hyperref}

% use normal fontstyle for urls; if not set, a tt-font is used
\urlstyle{same}

\pagestyle{headings}

% distance of line from head:
\headsep4mm

\iflatexml
%% nothing
\else
% recompute type area
\typearea[10mm]{current}		
\fi

% no indentation at the beginning of a new paragraph
%\setlength{\parindent}{0pt}

% new commands for math mode

% vector and absolute value
\newcommand{\abs}[1]{\left\lvert#1\right\rvert}
\renewcommand{\vec}[1]{\boldsymbol{#1}}

% vector operators
\DeclareMathOperator{\grad}{grad}
\DeclareMathOperator{\Div}{div}    %\div is used by amsmath
\DeclareMathOperator{\rot}{rot}

% brackets
\newcommand{\lra}[1]{ \left( #1 \right) }
\newcommand{\lrb}[1]{ \left[ #1 \right] }
\newcommand{\lrc}[1]{ \left\{ #1 \right\} }

% fractions
\newcommand{\td}[1]{\frac{d}{d #1}}
\newcommand{\ttd}[2]{\frac{d #2}{d #1}}
\newcommand{\pd}[1]{\frac{\partial}{\partial #1}}
\newcommand{\ppd}[2]{\frac{\partial #2}{\partial #1}}
\newcommand{\pdd}[1]{\frac{\partial^2}{\partial #1^2}}
\newcommand{\fa}{\frac{1}{a}}
\newcommand{\fh}{\frac{\dot{a}}{a}}
\newcommand{\ft}{\frac{1}{\sqrt{2\pi}}}
\newcommand{\fft}{\frac{1}{2\pi}}
\newcommand{\ffft}{\frac{1}{(2\pi)^{3/2}}}

% integrals
\newcommand{\iinf}{\int_{-\infty}^{\infty}}
\newcommand{\iiinf}{\iint_{-\infty}^{\infty}}
\newcommand{\iiiinf}{\iiint^{\infty}_{-\infty}}

% filtered quantities
%\newcommand{\fil}[1]{{<} #1 {>}}
\newcommand{\fil}[1]{\langle #1 \rangle}
\newcommand{\ffil}[1]{{\ll} #1 {\gg}}
\newcommand{\cfil}[1]{{\prec}#1{\succ}}
\newcommand{\chat}[1]{\accentset{\curlywedge}{#1}}
\newcommand{\ol}[1]{\overline{#1}}

\title{Mostly harmless notes}
\author{Andreas Maier}
\date{\today}
\begin{document}
\maketitle
\tableofcontents
\part{Introduction}
This book holds a collection of mostly harmless notes on mathematics and physics. The topics covered are simply all things I found interesting. I mostly cover topics which 
you won't find easily in Wikipedia or a usual text book. Often I also approach things from a different point of view (namely my point of view). My hope is that these notes grow into an interesting addition to the existing literature. 
\chapter{Mathematics}

\section{Numbers}

\subsection{Triangle inequality}

\subsubsection{Number inequality}

We start with the obvious inequality that the absolute value of a number
$\left|a\right|$ is always greater or equal the number $a$ itself 
\begin{align}
|a| \geq a
\end{align} 
Using this we can prove that for numbers $a$ and $b$ we have 
\begin{align}
|a| + |b|  \geq |a+b|
\end{align} 
because 
\begin{align}
(|a| + |b|)^2  &\geq (|a+b|)^2 \\
|a|^2  + 2|a||b| + |b|^2 &\geq |a^2 + 2ab + b^2| \\
2|a||b| &\geq 2|ab|
\end{align} 
The last line is always true if $|a| \geq a$, which completes our
prove. 

\subsection{Cauchy-Schwarz inequality} 
Because
$\sin^2(\phi) + \cos^2(\phi) = 1$ we have 
\begin{align}
|a|^2|b|^2 = |a|^2|b|^2 (\sin^2(\phi) + \cos^2(\phi))
\end{align} 
With the definition of the scalar product
$|a\cdot b| = |a||b|\cos(\phi)$ and the cross product
$|a\times b| = |a||b|\sin(\phi)$ of vectors $a$ and $b$ we can
write this expression as 
\begin{align}
|a|^2|b|^2 = |a\times b|^2 + |a\cdot b|^2
\end{align} 
This is called \href{https://en.wikipedia.org/wiki/Lagrange\%27s_identity}{Lagrange's identity}. Since all terms are squared and
therefor positive we immediatelly can derive the inequalities 
\begin{align}
|a|^2|b|^2 &\geq |a\cdot b|^2 \\
|a|^2|b|^2 &\geq |a\times b|^2
\end{align} 
The first of these equations is the Cauchy-Schwarz inequality. The
second equations doesn't seem to have a name in the literature.

The Cauchy-Schwarz inequality comes in many different forms. For example
for $n$-dimensional vectors it can be written in cartesian coordinates
like 
\begin{align}
\sum (a_i)^2 \sum (b_i)^2 \geq \left(\sum a_i b_i \right)^2
\end{align} 
This can even be generalized to uncountable infinite dimensional
vectors (also called continuous square integrable functions) like 
\begin{align}
\left(\int \left|a(x)\right|^2 dx\right) \cdot \left(\int \left|b(x)\right|^2 dx\right) \geq \left|\int a(x) \cdot b(x)dx\right|^2
\end{align}

\section{Matrices}

In the following we want to restrict our discussion to matrices with
real entries.

\subsection{Square matrices}
Every square matrix can be split into a symmetric and an antisymmetric
(skew-symmetric) part 
\begin{align}
A=\underbrace{\frac{1}{2}\lra{A + A^T}}_{\text{symmetric}}
+\underbrace{\frac{1}{2}\lra{A - A^T}}_{\text{antisymmetric}}
\end{align}


\subsection{Symmetric matrices}

A symmetric matrix is a square matrix, that is equal to it's transpose.
\begin{align}
A = A^T
\end{align} 
The sum of two symmetric matrices is again a symmetric matrix, but
the product of two matrices is in general not symmetric. The product of
two symmetric matrices $A$ and $B$ is symmetric only, if the two
matrices commute: 
\begin{align}
AB = (AB)^T \ \text{if}\ AB = BA
\end{align} 
Symmetric matrices with real entries have only real eigenvalues. That
is why in principle, every symmetric matrix is equivalent to a diagonal
matrix with its eigenvalues being the entries on the diagonal.

\subsection{Applications of Matrices}
\subsubsection{Solving unsolvable equations}

We start with a simple equation: 
\begin{align}
A^2 * 1 = 1 
\end{align} 
What is the transformation A, when applied twice, which turns 1 into
1 ? The answer is simple: $A=1$ or $A=-1$. But what is the
transformation $B$, when applied twice, which turns 1 into -1 ? 
\begin{align}
B^2 * 1 = -1 
\end{align} 
The answer it turns out, is not so simple. Based on the common rules
of multiplication for real numbers, there seems to be no way to solve
this equation for $B$.

However there is a way out of the dilemma. The reader might have
noticed, that we didn't call $A$ or $B$ a number or a variable, but
a transformation. We could have also called it also an operator. This
should be a hint, that maybe simple numbers are not enough to solve such
an equation.

So let's think a bit. Can this equation maybe interpreted in a
geometrical way? Let's imagine the line of numbers. What we want is a
way, to move or transform the point 1 to the point -1, but doing it with
two steps (One step would be simple, this could be reflection at the
zero point). If we draw the line on a sheet of paper and look from far
away the solution might be become more obvious: The line is embedded on
the two dimensional surface of the paper! So we might interpret our
source point 1 in fact as a vector (1,0) and our target point -1 as
vector (-1,0). Maybe this helps. Let's rewrite our equation in two
dimensions 
\begin{align}
B^2 * \begin{pmatrix} 1 \\ 0 \end{pmatrix} = \begin{pmatrix} -1 \\ 0 \end{pmatrix} 
\end{align} 
So what we want is a transformation $B$, which when used twice on
the vector (1,0) will turn the vector into the vector (-1,0). Vector
(-1,0) is pointing into the opposite direction of vector (1,0), so it is
basically rotated by 180 degrees. So we are searching for a
transformation which rotates a vector in two steps by 180 degrees. Now
it should be obvious that the solution for $B$ must be a
transformation, which rotates a vector by 90 degrees, in short
$B = R(90^{\circ})$ (or a rotation in the other direction by -90
degrees).

So we found a solution by geometrical intuition, let's try to make it
precise. From linear algebra we know, that a transformation which turns
a 2D vector into another 2D vector must be a 2x2 matrix. So let's
rewrite the equation as a matrix equation 
\begin{align}
\begin{pmatrix} a & b \\ c & d \end{pmatrix} * \begin{pmatrix} a & b \\ c & d \end{pmatrix} * \begin{pmatrix} 1 \\ 0 \end{pmatrix} = \begin{pmatrix} -1 \\ 0 \end{pmatrix} 
\end{align} 
If we compute the matrix product on the left hand side we get 
\begin{align}
\begin{pmatrix} a^2 + bc & b (a+d) \\ c(a+d) & d^2 + bc \end{pmatrix} \begin{pmatrix} 1 \\ 0 \end{pmatrix} = \begin{pmatrix} -1 \\ 0 \end{pmatrix} 
\end{align} 
Further simplifying we get two equations 
\begin{align}
a^2+bc = -1 &\ & ac + cd = 0
\end{align} 
To solve this equation one might attempt to set $c=0$. But in this
case we would end up where we started because the equation left would be
$a^2=-1$. So we have to assume $c \neq 0$ to find a sensible
solution. So we get 
\begin{align}
c = -\frac{a^2+1}{b} &\ & a = -d
\end{align} 
From these equation we see that we also have to assume $b \neq 0$.
But without loss of generality we can set $a=0$,$d=0$ and are left
with 
\begin{align}
bc = -1
\end{align} 
If we restrict ourself to integer numbers, we finally have two
solutions for our transformation 
\begin{align}
B = \begin{pmatrix} 0 & -1 \\ 1 & 0 \end{pmatrix}  &\ & B^* = \begin{pmatrix} 0 & 1 \\ -1 & 0 \end{pmatrix}
\end{align} 
So what seemed impossible to solve in 1D with simple numbers turned
out to have quite simple solutions in 2D in the form of 2x2 matrices.

\subsubsection{Solving polynomial equations}
A polynomial equation of order $n$ is an equation that look like 
\begin{align}
a_nx^n + a_{n-1}x^{n-1} + \cdots + a_1 x + a_0 = 0
\end{align} 
Solving these kind of equations has been a hobby of mathematicans
since the invention of math. But whereas quadratic equations ($n=2$)
could be solved since ancient times, finding a general solution for
cubic ($n=3$) and quartic ($n=4$) equations turned out to be much
harder. Rafael Bombelli made a crucial step in 1572 when - in a
desparate move - he invented complex numbers to solve cubic equations,
which had been unsolvable up to that time. Had he known matrices and
linear algebra, the invention of ``complex numbers'' would have been
unnecessary.

This is so, because there is a deep connection between polynominal
equations and matrices. Actually it turns out that \textbf{any
polynomial with degree $n$ is the characteristic polynomial of some
\href{https://en.wikipedia.org/wiki/Companion_matrix}{companion matrix}
of order $n$}. So the problem of solving a polynominal equation is
equivalent to solving the characteristic equation of the companion
matrix. But solving the characteristic equation of a matrix means
computing the eigenvalues of that matrix. And computing eigenvalues has
a simple geometric meaning: They give the factor by which an eigenvector
(a vector which direction is left unchanged by the matrix
transformation) is stretched by the matrix transformation. The
eigenvalues are the scale factors of the linear transformation
represented by the matrix. This means that solving a polynomial equation
is equivalent to computing the scale factors of a corresponding linear
transformation.

Knowning this we can nowadays understand, why some polynominal equations
have no solutions. These are the equations, which correspond to
matrices, which don't have eigenvalues, meaning the transformation
doesn't leave any vector unchanged. From linear algebra we know, that
these transformations describe rotations. And this is the connection to
the 2d rotation matrix we found in the previous chapter. The polynominal
equation 
\begin{align}
x^2 = -1
\end{align} 
has no solution, because it corresponds to a matrix describing a 90
degree rotation.

On the other side the
\href{https://en.wikipedia.org/wiki/Fundamental_theorem_of_algebra}{fundamental
theorem of algebra} is easy to understand, when thinking of polynomials
as being represented by matrices. The number of solutions to a
polynomial of degree $n$ is the same as the number of eigenvalues of
the corresponding companion matrix.

\section{Complex Numbers}
Complex numbers are not numbers. They cannot be ordered according to their size. This basic insight makes clear that trying to work with complex numbers like with usual “real numbers” must fail (e.g. division doesn’t work) and in general is also the reason for the big confusion around them.

But complex numbers are also no vectors (in the geometrical sense). The multiplication rule for complex numbers is completely different from the usual scalar product of geometrical vectors. Multiplying two complex numbers yields another complex number, whereas the usual scalar multiplication of geometrical vectors yields a scalar. So although a complex number can be represented as a set of two numbers, this set of two numbers should not(!) be visualized as geometrical vector.

Instead the modern view is that complex numbers are 2D matrices. They represent the group of (antisymmetric) 2D rotation matrices (\url{https://en.wikipedia.org/wiki/Complex_number#Matrix_representation_of_complex_numbers}). All the features of complex numbers follow naturally from this representation, e.g. multiplying two matrices yields another matrix. Unfortunately most textbooks don’t even mention the matrix representation of complex numbers, although this really makes clear what complex “numbers” are, how they can be extended (e.g. quaternions are 3D rotation matrices) and how they fit into the bigger picture which is \url{https://en.wikipedia.org/wiki/Group_theory}.


\section{Basic probability theory}


\subsection{Definition of probability}

An experiment can measure if an event $A$ happend or not. If we repeat
an experiment $n$ times and we measure that the event $A$ happened
$m$ times we define the probability that the event $A$ happens as
\begin{align}
P(A) = \lim\limits_{n \to \infty} \frac{m}{n}
\end{align}

If event $C$ means that event $A$ \emph{or} event $B$ can happen
we write $C = A \cup B$. If event $D$ means that event $A$
\emph{and} event $B$ happen, we write $D = A \cap B$. Classical
probability theory is then based on the following three axioms (called
the Kolmogorov axioms):

\begin{enumerate}
\item
  Every event $A$ has a real non-negative probability $P(A) \ge 0$.
\item
  The probability that any event from the event space will happen is
  one: 
  
  $P(A \cup A^c) = 1$ (where $A^c$ is the complement event to
  $A$ in the event space)
\item
  The probabilities of mutually exclusive events ( $P(A \cap B) = 0$ )
  add: 
  
  $P(A \cup B) = P(A) + P(B)$
\end{enumerate}

From the last axiom it also follows that in general
\begin{align}
P(A \cup B) = P(A) + P(B) - P(A \cap B)
\end{align}
Although these axioms seem unavoidable it should be mentioned that
quantum probability theory violates axiom 1 and axiom 3. Axiom 3 is
violated, because measurement of events in quantum mechanics (QM) are
not commutative, meaning the measurement of event A often must influence
the measurement of event B. Axiom 1 is violated since QM must describe
interference effects between events and does this by introducing
\href{https://en.wikipedia.org/wiki/Negative_probability}{negative
probabilities} (To be more precise, the probability wave function of QM
is complex, because in the theory one must basically take the square
root of the negative probabilities). But as Dirac put it: ``Negative
probabilities should not be considered nonsense. They are well defined
concepts mathematically, like a negative sum of money. Negative
probabilities should be considered simply as things which do not appear
in experimental results.''

So it is possible to work with things like negative or complex
probabilities. But to be able to derive the central limit theorem it is
necessary that the three axioms of Kolmogorov for classical probability
hold.

\subsection{Random variables}
Examples for random variables are e.g.~the number on a thrown dice, the
lifetime of a instable radioactive nucleus, the amplitude of
athmospheric noise recorded by normal radio. In general a random
variable $X$ takes finite real values $x$ where the probability that
$X$ takes the value $x$ in a given intervall from $a$ to $b$
depends on the event described by $X$. We write 
\begin{align}
P(a < X < b) = \int_a^b f_X(x) dx 
\end{align} 
where $f_X(x)$ is the so called probability density function
characteristic for the event. We have to be careful to distinguish the
random variable $X$ from its value $x$ it takes after the
measurement. If $a < x < b$ then the probability $P(a < x < b)$ is
always one, because $x$ is just a number between $a$ and $b$. But
the value of $P(a < X < b)$ depends on the form of the probability
density function (Note however that $P(-\infty < x < \infty)$ =
$P(-\infty < X < \infty)$ according to the second axiom). So a random
variable is - despite its name - actually not a number or value, it is
\href{https://www.mathsisfun.com/data/random-variables.html}{a set of
possible values from a random experiment}, where each value has a
probability associated with it. To describe a experiment of throwing a
dice one could write 
\begin{align}
X = \{(1,1/6),(2,1/6),(3,1/6),(4,1/6),(5,1/6),(6,1/6)\}
\end{align} 
where the first value of each tuple is the possible outcome $x_i$
(called \href{https://en.wikipedia.org/wiki/Random_variate}{Random
Variate}), the second is the corresponding probability $p_i$. The
probability is 
\begin{align}
P(a < X < b) = \sum_{a < x_i < b} p_i  
\end{align}

or in case that the possible outcomes are a continous set $x(t)$ with
the corresponding
\href{https://en.wikipedia.org/wiki/Probability_density_function}{probability
density function} $p(t)$ 
\begin{align}
P(a < X < b) = \int_a^b p(t) dt  
\end{align}

In the special case $a=-\infty$ the probability $P$ only depends on
the upper limit $b$ 
\begin{align}
P(-\infty < X < b) = P(X < b) = \int_{-\infty}^b p(t) dt = F(b)  
\end{align} 
and we call $F(b)$ the
\href{https://en.wikipedia.org/wiki/Cumulative_distribution_function}{cumulative
distribution function}.

It should be noted that a possible outcome or random variate $x_i$ is
in principle a functional $x[p(t)]$ over the space of probability
density functions. This can be seen in the way how pseudo random numbers
are generated on a computer. One usually has a
\href{https://en.wikipedia.org/wiki/Pseudorandom_number_generator}{random
number generator} (often also called deterministic random number
generator, because starting from the same seed, it will generate the
same series of random numbers) generating uniformly distributed numbers
between 0 and 1. These uniformly distributed numbers are then
transformed to non-uniform random numbers by methods like
\href{https://en.wikipedia.org/wiki/Inverse_transform_sampling}{inverse
transform sampling} or
\href{https://en.wikipedia.org/wiki/Rejection_sampling}{rejection
sampling}, which basically make use of the probability density function
$p(t)$. So the scalar real value taken by a random variable generated
by a computer depends on the form of a function, making $x[p(t)]$ a
functional and the random variable $X$ a set of functionals.

\subsection{Multiple random variables}

Assume you throw one red and one blue dice in an experiment. The number
on the red dice would be random variable $X$, the number of the blue
dice random variable $Y$. Under normal circumstances the number on
each dice would be independent of each other. We say, the random
variables $X$ and $Y$ are uncorrelated. However, assume that the
dice are magnetic. In that case the number shown on each dice might not
be independent anymore.

To describe such an experiment where we measure a pair of random
variables we use a joint probability and a joint distribution function:
\begin{align}
P(a < X < b, c < Y < d) = \int_a^b\int_c^d f_{XY}(x,y) dx dy 
\end{align} 
In analogy we can also describe experiments with $n$ random
variables by using a $n$-dimensional joint distribution function (In
the limit that $n$ would approach an uncountable infinity the
probability would be expressed by an infinite dimensional integral, see
also
\href{https://en.wikipedia.org/wiki/Functional_integration}{Functional
integration})

Quantum mechanics = asymmetric covariance matrix??

\section{More topics}
\subsection{Linear algebra}

\begin{itemize}
\item \url{https://physics.stackexchange.com/questions/35562/is-a-1d-vector-also-a-scalar}
\item \url{https://math.stackexchange.com/questions/219434/is-a-one-by-one-matrix-just-a-number-scalar}
\item \url{https://en.wikipedia.org/wiki/Eigenvalues_and_eigenvectors#Eigenvalues_of_geometric_transformations}
\end{itemize}
\subsubsection{Functions of matrices}
\begin{itemize}
\item \url{https://en.wikipedia.org/wiki/Matrix_function}
\item \url{https://en.wikipedia.org/wiki/Matrix_exponential}
\item \url{https://en.wikipedia.org/wiki/Logarithm_of_a_matrix}
\item \url{https://math.stackexchange.com/questions/1149598/how-to-solve-a-non-linear-matrix-equation-over-integer-numbers}
\end{itemize}
\subsection{Transformations and groups}

\begin{itemize}
\item Prove of eulers formula as a solution to 2d wave equation: \url{http://math.stackexchange.com/a/3512/27609}
\item \url{https://en.wikipedia.org/wiki/Linear_canonical_transformation}
\item \url{https://en.wikipedia.org/wiki/Hartley_transform}
\item \url{https://en.wikipedia.org/wiki/Split-complex_number}
\item \url{https://en.wikipedia.org/wiki/Dual_number} (\url{https://math.stackexchange.com/questions/1120720/are-dual-numbers-a-special-case-of-grassmann-number})
\item \url{https://en.wikipedia.org/wiki/Grassmann_number} \href{https://math.stackexchange.com/questions/1108045/relationship-between-levi-civita-symbol-and-grassmann-numbers}{Grassmann vectors}
\item \url{https://en.wikipedia.org/wiki/Quaternion} (\url{https://math.stackexchange.com/questions/147166/does-my-definition-of-double-complex-noncommutative-numbers-make-any-sense})
\item \url{https://math.stackexchange.com/questions/2083950/relationship-between-levi-civita-symbol-and-complex-quaternionic-numbers}
\end{itemize}
\subsection{Series}

\begin{itemize}
\item \url{http://blog.wolfram.com/2014/08/06/the-abcd-of-divergent-series}
\begin{itemize}
\item \url{http://physicsbuzz.physicscentral.com/2014/01/redux-does-1234-112-absolutely-not.html}
\item \url{https://www.quora.com/Whats-the-intuition-behind-the-equation-1+2+3+-cdots-tfrac-1-12}
\end{itemize}
\end{itemize}

\subsection{Calculus}

\subsubsection{Euler-MacLaurin}
\begin{itemize}
\item \url{https://people.csail.mit.edu/kuat/courses/euler-maclaurin.pdf}
\item \url{http://www.hep.caltech.edu/~phys199/lectures/lect5_6_ems.pdf}
\item \url{https://terrytao.wordpress.com/2010/04/10/the-euler-maclaurin-formula-bernoulli-numbers-the-zeta-function-and-real-variable-analytic-continuation}
\end{itemize}

\subsubsection{Watsons Triple Integrals}
\begin{itemize}
\item \url{http://mathworld.wolfram.com/WatsonsTripleIntegrals.html}
\item \url{http://www.inp.nsk.su/~silagadz/Watson_Integral.pdf}
\end{itemize}

\subsubsection{Generalized Calculus}
\begin{itemize}
\item \url{https://en.wikipedia.org/wiki/Product_integral}
\item \url{http://math2.org/math/paper/preface.htm}
\item \url{http://www.gauge-institute.org/calculus/PowerMeansCalculus.pdf}
\end{itemize}

\subsubsection{Finite calculus}
\begin{itemize}
\item \url{https://www.cs.purdue.edu/homes/dgleich/publications/Gleich\%202005\%20-\%20finite\%20calculus.pdf}
\item \url{https://en.wikipedia.org/wiki/Concrete_Mathematics}
\end{itemize}

\subsubsection{Iterative roots and fractional iteration}
\begin{itemize}
\item \url{http://reglos.de/lars/ffx.html}
\item \url{https://mathoverflow.net/questions/17605/how-to-solve-ffx-cosx}
\end{itemize}

\subsection{Geometry}
\begin{itemize}
\item \href{https://www.friedrich-verlag.de/fileadmin/redaktion/sekundarstufe/Mathematik/Der_Mathematikunterricht/Leseproben/Der_Mathematikunterricht_3_13_Leseprobe_2.pdf}{Cutting a cube along the diagonal}
\item \url{https://en.wikipedia.org/wiki/Visual_calculus}
\end{itemize}

\subsection{Weird constants and functions}

\subsubsection{Euler-Mascheroni constant}
\begin{itemize}
\item \url{https://en.wikipedia.org/wiki/Euler\%E2\%80\%93Mascheroni_constant#Generalizations}
\end{itemize}

\subsubsection{Universal Parabolic constant}
\begin{itemize}
\item \url{https://en.wikipedia.org/wiki/Universal_parabolic_constant}
\item \url{http://mathworld.wolfram.com/UniversalParabolicConstant.html}
\item \url{https://mathoverflow.net/questions/37871/is-it-a-coincidence-that-the-universal-parabolic-constant-shows-up-in-the-soluti}
\end{itemize}

\subsubsection{Apery's constant}
\begin{itemize}
\item \url{https://en.wikipedia.org/wiki/Ap%C3%A9ry%27s_constant}
\item \url{https://math.stackexchange.com/questions/12815/riemann-zeta-function-at-odd-positive-integers/12819#12819}
\end{itemize}

\subsubsection{Gauss's constant}
\begin{itemize}
\item \url{https://en.wikipedia.org/wiki/Gauss\%27s_constant}
\item \url{https://en.wikipedia.org/wiki/Lemniscatic_elliptic_function}
\item \url{https://en.wikipedia.org/wiki/Particular_values_of_the_Gamma_function}
\end{itemize}

\subsubsection{Riemann Zeta function}
\begin{itemize}
\item \url{https://math.stackexchange.com/questions/1792755/connection-between-the-area-of-a-n-sphere-and-the-riemann-zeta-function}
\item \url{https://suryatejag.wordpress.com/2011/11/24/riemann-functional-equation-and-hamburgers-theorem}
\end{itemize}


\subsection{Probability}
\begin{itemize}
\item \url{https://en.wikipedia.org/wiki/Secretary_problem}
\item \url{https://en.wikipedia.org/wiki/Kelly_criterion}
\end{itemize}

\subsubsection{Sample size}
\begin{itemize}
\item \url{https://stats.stackexchange.com/questions/192199/derivation-of-formula-for-sample-size-of-finite-population/192601#192601}
\item \url{https://math.stackexchange.com/questions/926478/how-does-accuracy-of-a-survey-depend-on-sample-size-and-population-size/1357604#1357604}
\item \url{https://onlinecourses.science.psu.edu/stat414/node/264}
\item \url{http://www.surveysystem.com/sscalc.htm}
\item \url{http://research-advisors.com/tools/SampleSize.htm}
\end{itemize}



\part{Physics}
\chapter{Classical mechanics}
In the view of classical mechanics the world consists of point particles with constant mass $m$, 
position $\vec{x(t)}$ and velocity $\vec{v(t)}$. The number of particles is countable 
(not an uncountable infinity) and stays constant. The particles don't split or unite. 
The time is a global parameter which is the same for each particle. This description 
of the world is sometimes also called point mechanics.

It is amazing how many phenomenoms one can describe with this simple model of the world.
On the other side this short description already hints to the several limitations of this model. 
The standard model of physics nowadays based on quantum field theory basically abandons
all the assumptions made by classical mechanics. 

\chapter{Fluid dynamics}
\section{Introduction}
Fluids
\footnote{Fluids are materials which behave like a liquid, so they can be
deformed by shear stresses without limits. All gases and liquids are fluids.}
 in fluid dynamics are treated as continuous fields. Each point of the
field represents a fluid element consisting of several point particles.
\footnote{This is not a contradiction. Because the molecules of a fluid
are treated as pointlike particles (which have no volume) one can have a bunch
of point particles at one point.}
The statistical behaviour of these point particles 
defines quantities like density, temperature, pressure and average velocity
for each fluid element. In this sense an fluid can be represented by a number of
contiuous fields (density, velocity, pressure, temperature,...)
which defines at every point a certain statistical quantity of a bunch 
of fluid molecules.

\subsection{Substantial derivative}
To compute the change with time of a scalar quantity $A(x,y,z,t)$ at a fixed
point in
space $(x,y,z)$, we get
 \begin{align}
\td{t} A=\pd{t} A.
\end{align}
However in fluid dynamics we are often interested in the temporal change of a
quantity in a certain local fluid element, which moves with the fluid. This
means
$A=A(x(t),y(t),z(t),t)$ and therefore we get for the change
\begin{align}
\frac{dA}{dt}=\frac{dA}{dx}\frac{dx}{dt}+\frac{dA}{dy}\frac{dy}{dt}+
\frac{dA}{dz}\frac{dz}{dt}+\frac{\partial A}{\partial t}.
\end{align}
So if in general we want to express the total derivative $\frac{d}{dt}$ by
quantities at fixed space points, we can make use of the so called substantial
derivative
\begin{align}
\td{t}=\pd{t} + v_j \pd{r_j}. \label{eq:1}
\end{align}
\subsection{Reynolds transport theorem}
If $A=\int_{V(t)} \alpha(\vec{r},t) dV$ is a scalar quantity which is conserved
in a local fluid element moving with the fluid (and therefore having a time
dependent
volume) we can write
\begin{align}
\frac{d}{dt} A = \frac{d}{dt} \int_{V(t)} \alpha(\vec{r},t) dV = 0.
\end{align}
But because the boundary of the integral is time dependent, we cannot exchange
integration with the time derivative. Therefore we have to ascribe the
integration over the time dependent volume $V(t)$ to the volume $V_0$ at time
$t=0$. The transformation of the volume element $dV_0$ (at time $t=0$) to the
the volume element $dV_0$ can be described by
\begin{align}
dV=J dV_0 &&\text{mit } J= \abs{\frac{\partial(x,y,z)}{\partial (x_0,
y_0,z_0)}},
\end{align}
where $J$ is called Jacobian determinant \footnote{Also see Appendix
\ref{jacobi}.}. It describes the change of the fluid element if it is
transported with the fluid. So we can express $\frac{d}{dt} A$ by
\begin{align}
\frac{d}{dt} A = \int_{V_0} \frac{d}{dt}(\alpha(\vec{r},t) J) dV_0
= \int_{V_0} \left(J \frac{d\alpha}{dt}+\alpha \frac{dJ}{dt}\right) dV_0.
\end{align}
Using the substantial derivative and $\frac{dJ}{dt}=J \pd{r_j} v_j$
\footnote{For a derivation see Appendix \ref{jacdt}.} leads to
\begin{align}
\td{t} A = \int_{V_0} \left[\frac{\partial \alpha}{\partial t} + 
(v_j \pd{r_j}) \alpha + \alpha (\pd{r_j} v_j)\right] J  dV_0.
\end{align}
From this we get the Reynolds transport theorem
\begin{align}
\td{t} A = \int_{V(t)} \left[\pd{t} \alpha + 
\pd{r_j} (v_j \alpha)\right]  dV. \label{eq:RT}
\end{align}
Because it is valid for arbitrary volumes we can also write  it as a generalised
continuity equation
\begin{align}
\pd{t}(\alpha)+ \pd{r_j} (v_j \alpha) = 0.
\label{eq:RTdiff}
\end{align}
\section{General compressible fluid}
\subsection{Balance equations}
\subsubsection{Mass equation}
The mass inside a local fluid element $M=\int_V \rho(r_i) dV$ is conserved.
Therefore we get the balance equation for the mass from the generalised 
continuity equation setting $\alpha=\rho$ in the form
\begin{align}
\pd{t}\rho + \pd{r_j}(v_j \rho) = 0. \label{eq:3}
\end{align}
\subsubsection{Momentum equation}
The momentum $P_i$ inside a local fluid element is $P_i=\int_V \rho(r_i) v_i
dV$. The change of the momentum with time $\frac{d}{dt} P_i$ is equal to the sum
of the forces on the fluid element
\begin{align}
\td{t} P_i=F_i=F_{p,i}+F_{visc,i}+F_{g,i}.
\end{align}
Here we have restricted ourself to a viscous, selfgravitating fluid, so the sum 
of forces on each fluid element consists of 
\begin{itemize}
\item the thermodynamic pressure on the surface $A$ of the fluid element
\begin{align}
F_{p,i}= - \oint_A p n_i dA = - \int_V \pd{r_i} p dV,
\end{align}
\item the viscous force meaning the irreversible transfer of momentum due to
friction between the surfaces of the fluid elements 
\begin{align}
F_{visc,i}= \oint_A \sigma'_{ij} n_j dA = \int_V \pd{r_j} \sigma'_{ij} dV,
\end{align}
\item the gravitational force, where for a selfgravitationg fluid $g_i$ is 
generated by the fluid itself (not only by the local fluid element, because
gravity is a long-range force)
\begin{align}
F_{g,i}= \int_V \rho g_i dV.
\end{align}
\end{itemize}
Using the generalized continuity equation \eqref{eq:RTdiff} we can express the
temporal change of each component of the momentum of a fluid element
setting $\alpha=\rho v_i$ in the form
\begin{align}
\pd{t}(\rho v_i) + \pd{r_j}(v_j \rho v_i) = -\pd{r_i}p + \pd{r_j}\sigma'_{ij}
+\rho g_i.
\end{align}
With the help of the continuity equation \eqref{eq:3} we can write the momentum
equation in the often used form called Euler equation
\begin{align}
\pd{t}(v_i) + v_j \pd{r_j}( v_i) = -\frac{1}{\rho}\pd{r_i}p +
\frac{1}{\rho}\pd{r_j}\sigma'_{ij} + g_i. \label{eq:vel}
\end{align}
\subsubsection{Kinetic energy equation}
If we multiply equation \eqref{eq:vel} with the velocity $v_i$, use
$\frac{1}{2}\frac{d x^2}{dt} = x \frac{dx}{dt}$  and the continuity
equation we get an equation for the kinetic energy of a local fluid element in
conservation form
\begin{align}
\pd{t}(\frac{1}{2}\rho v^2) + \pd{r_j}(v_j\frac{1}{2}\rho v^2) =
-v_i\pd{r_i}p + v_i\pd{r_j}\sigma'_{ij}+v_i\rho g_i
\end{align}
This equation shows us that locally the kinetic energy is not conserved
(otherwise the right-hand side of the equation should be zero).
\subsubsection{Internal energy equation}
If we assume that each fluid element is in thermal equilibrium the first law of
thermodynamics does hold locally and we can write for the internal
energy of a fluid element\footnote{Why is there no gravitational effect on the 
internal energy? Can self-gravity  be understood similar to 
Van-der-Waals forces? This would imply that the pressure of an ideal gas has 
to be reduced by some factor $p=\frac{NkT}{V}-\frac{a}{V^2}, a \sim G$.}
 
\begin{align}
E_{int} = \int dE_{int} = \int T dS - \int p dV = \int \rho T s dV - \int p dV
\end{align}
with $s=\frac{dS}{dm}$ and $\rho=\frac{dm}{dV}$. Because $T$ and $p$ are
understood as the average temperature and pressure in the fluid element, they can 
be moved out of the integral so that
\begin{align}
E_{int} = \int \rho e_{int} dV = T \int \rho s dV - p \int dV.
\end{align}
with $e_{int}=\frac{dE_{int}}{dm}$.
If we take the time derivative of the internal energy we get
\begin{align}
\begin{split}
\td{t} \int \rho e_{int} dV &= \td{t}\lra{T \int \rho s dV}-\td{t}\lra{p \int
dV} \\
&= T \td{t} \int \rho s dV - p \td{t} \int dV + \int \rho s dV \td{t}T 
- \int dV \td{t} p
\end{split}
\end{align}
Because we assume local thermodynamic equilibrium $\td{t} T=0$ and $\td{t} p=0$
and the last two terms on the right hand side vanish. The other terms can be
computed by using the Reynolds transport theorem and we get the balance
equation for the internal energy of a fluid element
\begin{align}
\pd{t} \rho e_{int} + \pd{r_j} v_j \rho e_{int} = T \lra{\pd{t} \rho s +
\pd{r_j} v_j \rho s} -p \pd{r_j} v_j \label{eq:eint}
\end{align}

\subsubsection{Global dissipation of kinetic energy}
For investigating the conservation of the kinetic energy of the
whole fluid we write using the Reynolds transport theorem \eqref{eq:RT}
\begin{align}
\td{t} E_{kin} = \td{t} \int_V \frac{1}{2}\rho v^2 dV = 
\int_V \pd{t} \lra{\frac{1}{2}\rho v^2} + 
\pd{r_j} \lra{v_j \frac{1}{2}\rho v^2} dV
\end{align}
The second term on the right hand side can be transformed with Gauss's theorem
to an integral over the surface of the whole fluid
\begin{align}
\int_V \pd{r_j} \lra{v_j \frac{1}{2}\rho v^2} dV = 
\oint_A v_j \frac{1}{2}\rho v^2 dA = 0.
\end{align}
The surface integral is zero because the velocity on the boundary of the fluid
$v_j = 0$. Therefore we can write
\begin{align}
\td{t} E_{kin}=\pd{t} E_{kin}=\int_V \pd{t} \lra{\frac{1}{2}\rho v^2} dV =
\int_V \rho v_i \pd{t} v_i + \frac{1}{2} v^2 \pd{t}\rho dV 
\end{align}
Inserting the Euler equation \eqref{eq:vel} and using the continuity equation
\eqref{eq:3} and $g_i=\pd{r_i}\phi$ we can transform this to
\begin{align}
\begin{split}
\pd{t} E_{kin} =& - \int_V \pd{r_j}\lrb{v_j \rho
\lra{\frac{1}{2}v^2+\frac{p}{\rho}+\phi} + v_i \sigma'_{ij}} dV\\ 
&+ \int_V p \pd{r_j} v_j dV - \int_V \sigma'_{ij} \pd{r_j} v_i dV
- \int_V \phi \pd{t}\rho dV.
\end{split}
\end{align}
The first term on the right hand side can be transformed with Gauss's theorem
to a surface integral. Again this surface integral is zero because on the
surface of the fluid the velocity $v_i,v_j=0$. So we are left with
\begin{align}
\pd{t} E_{kin} = \int_V p \pd{r_j} v_j dV - \int_V \sigma'_{ij}
\pd{r_j} v_i dV
- \int_V \phi \pd{t}\rho dV,
\end{align}
which is the generalization of equation (16.2) from \citet{Landau1991}
for general compressible fluids. So we see that the total kinetic energy for an
ideal, compressible fluid is not conserved.  

If we substitute the first term on the right hand side with the balance equation
for internal energy \eqref{eq:eint} and again makes use of Gauss's theorem we
get the following expression
\begin{align}
\pd{t} E_{kin} = \pd{t} \int_V \rho s dV -\pd{t} \int_V \rho e_{int} dV
- \pd{t} \int_V\rho \phi dV + \int_V \rho \pd{t} \phi dV 
- \int_V \sigma'_{ij} \pd{r_j} v_i dV \label{eq:glodis}
\end{align}
One might be tempted to the following conclusion, that
\begin{align}
\pd{t} E_{tot} = 
\pd{t} E_{kin}+\pd{t} E_{int}+\pd{t} E_{pot} = 
T \pd{t} S - \int_V \sigma'_{ij} \pd{r_j} v_i dV + \int_V \rho \pd{t} \phi dV 
\end{align}
If then one requires that the total energy should be constant, one would come
to the conclusion
\begin{align}
\pd{t} S = \frac{1}{T}\int_V \sigma'_{ij} \pd{r_j} v_i dV 
- \frac{1}{T} \int_V \rho \pd{t} \phi dV
\end{align}
This would lead to the statement that the total entropy of an ideal fluid
($\sigma'_{ij}=0$) is not constant but dependent on the time derivative of the
potential $\phi$. But this is not true!

To get the right answer first notice that the potential energy of a
selfgravitating system is not $\int \rho \phi dV$ but $\frac{1}{2}\int \rho
\phi dV$. So in the statement above the definition of the total energy was
wrong. The potential of a certain density distribution is the solution of the
poisson equation
\begin{align}
\phi(x_i,t)=-G \int_V \frac{\rho(x_j,t)}{\abs{x_i-x_j}} dV_j
\end{align}
and therefore the potential energy of a selfgravitating system can be expressed
as a double integral
\begin{align}
E_{pot}=-\frac{G}{2}\iint \frac{\rho(x_i,t)\rho(x_j,t)}{\abs{x_i-x_j}}dV_j dV_i.
\end{align}
If we now take the time derivative of this expression for the potential energy
of a selfgravitating system
\begin{align}
\begin{split}
\pd{t}E_{pot}&=
-\frac{G}{2} \iint \rho_i \frac{\pd{t}\rho_j}{\abs{x_i-x_j}}dV_j dV_i
-\frac{G}{2} \iint \pd{t}(\rho_i)\frac{\rho_j}{\abs{x_i-x_j}}dV_j dV_i\\
&=
-\frac{G}{2} \iint \rho_i \frac{\pd{t}\rho_j}{\abs{x_i-x_j}}dV_j dV_i
-\frac{G}{2} \iint \rho_j \frac{\pd{t}\rho_i}{\abs{x_i-x_j}}dV_i dV_j\\
&=
-\frac{G}{2} \iint \rho_i \frac{\pd{t}\rho_j}{\abs{x_i-x_j}}dV_j dV_i
-\frac{G}{2} \iint \rho_i \frac{\pd{t}\rho_j}{\abs{x_j-x_i}}dV_j dV_i\\
&=
-G \iint \rho_i \frac{\pd{t}\rho_j}{\abs{x_i-x_j}}dV_j dV_i \\
&=
\int \rho \pd{t} \phi dV
\end{split}
\end{align}
Here we used the abbreviation $\rho_i=\rho(x_i,t)$ and $\rho_j=\rho(x_j,t)$.
\footnote{It is also important to note, that even for a timedependent
potential the Greenfunction $\frac{1}{\abs{x_i-x_j}}$
is not timedependent!}

So we see that for a selfgravitating system
\begin{align}
\int \rho \pd{t} \phi dV = \pd{t} \lra{\frac{1}{2}\int \rho\phi dV}.
\end{align}
If we use this expression in \eqref{eq:glodis} and use $E_{pot}= \frac{1}{2}\int
\rho \phi dV$ we get
\begin{align}
\pd{t} E_{tot} = 
\pd{t} E_{kin}+\pd{t} E_{int}+\pd{t} E_{pot} = 
T \pd{t} S - \int_V \sigma'_{ij} \pd{r_j} v_i dV
\end{align}
Demanding that the total energy of the whole fluid should be conserved leads to
the right expression for the time evolution of the total entropy
\begin{align}
\pd{t} S = \frac{1}{T}\int_V \sigma'_{ij} \pd{r_j} v_i dV
\end{align}
So total entropy is conserved for an ideal fluid, even if we take selfgravity
into account\footnote{Nevertheless Penrose \citep{Penrose1989} pointed out,
that the entropy of a selfgravitating fluid should rise up to the point when
all matter is collapsed to a black hole, see also Appendix \ref{entro}.}.

\subsubsection{Local dissipation of kinetic energy}
For analyzing the dissipation of kinetic energy in a local fluid element we
have to make the same calculations as we did to get equation \eqref{eq:glodis},
but without the assumption of $v=0$ on the boundary. Doing this we arrive at
\begin{align}
\begin{split}
\td{t} E_{kin} =& \int_V \left[ \pd{r_j}\lra{v_i \sigma'_{ij}-v_j p} 
+ T\lra{\pd{t}\rho s + \pd{r_j} v_j \rho s} 
- \lra{\pd{t}\rho e_{int} + \pd{r_j} v_j \rho e_{int}} \right. \\
&\left. -\lra{\pd{t}\rho \phi + \pd{r_j} v_j \rho \phi}
+ \rho\pd{t}\phi -\sigma'_{ij}\pd{r_j}v_i \right] dV.
\end{split}
\end{align}
If we identify the second, third and fourth term on the right hand side with the
total time derivative of the entropy, the internal energy and the potential
energy respectively we get
\begin{align}
\td{t} E_{kin} + \td{t} E_{int} + \td{t} E_{pot} = 
\int_V \pd{r_j}\lra{v_i \sigma'_{ij}-v_j p} + T \td{t} S
+ \rho\pd{t}\phi -\sigma'_{ij}\pd{r_j}v_i dV.
\end{align}
If we additionally assume that locally the same entropy equation holds as
globally
\begin{align}
\td{t} S = \int_V \sigma'_{ij}\pd{r_j}v_i dV
\end{align}
or in differential form
\begin{align}
\pd{t}\rho s + \pd{r_j} v_j \rho s = \sigma'_{ij}\pd{r_j}v_i,
\end{align}
we are led to the following balance equation for the total energy
$e_{tot}=e_k+e_{int}+\phi$ for a local fluid element
\begin{align}
\pd{t}\rho e_{tot} + \pd{r_j} v_j \rho e_{tot} = -\pd{r_j}\lra{v_j p}
+\pd{r_j}\lra{v_i \sigma'_{ij}} + \rho \pd{t} \phi. \label{eq:etotal}
\end{align}
which is basically the sum of the three balance equations
\begin{align}
\pd{t}\rho \phi + \pd{r_j} v_j \rho \phi &=  + \rho \pd{t} \phi \\
\pd{t}\rho e_{k} + \pd{r_j} v_j \rho e_{k} &= -v_j \pd{r_j} p
+ v_i \pd{r_j} \sigma'_{ij}\\
\pd{t}\rho e_{int} + \pd{r_j} v_j \rho e_{int} &= -p \pd{r_j}v_j
+\sigma'_{ij} \pd{r_j}v_i
\end{align}
If we do not include potential energy in the total energy but use instead
$e = e_k+e_{int}$ we can write
\begin{align}
\pd{t}\rho e + \pd{r_j} v_j \rho e = -\pd{r_j}\lra{v_j p}
+\pd{r_j}\lra{v_i \sigma'_{ij}} - v_i \rho \pd{r_i} \phi. \label{eq:etotal2}
\end{align}
which can be split into two balance equations
\begin{align}
\pd{t}\rho e_{k} + \pd{r_j} v_j \rho e_{k} &= -v_j \pd{r_j} p
+ v_i \pd{r_j} \sigma'_{ij} - v_i \rho \pd{r_i} \phi\\
\pd{t}\rho e_{int} + \pd{r_j} v_j \rho e_{int} &= -p \pd{r_j}v_j
+\sigma'_{ij} \pd{r_j}v_i
\end{align}
where gravity is now just treated as a source in the balance equation of the
kinetic energy.

One might recognize, that we didn't mention selfgravity. In fact for the
derivation of the energy equation \eqref{eq:etotal} we assumed, that the
potential energy of the local fluid element is not due to selfgravity but due
to some external potential generated by the rest of the fluid. This assumption
is valid in case the local fluid element does not contribute much to the global
potential $\phi$. This means, that 
\begin{align}
\phi_{local}= -G \int_{V_{local}}\frac{\rho\lra{x_j}}{\abs{x_i-x_j}}dV_j \ll
\phi_{global}= -G \int_{V_{global}}\frac{\rho\lra{x_j}}{\abs{x_i-x_j}}dV_j
\end{align}
Because we know from the last chapter that the term $- v_i \rho \pd{r_i} \phi$
transforms into $-\frac{1}{2}\pd{t}\rho\phi$ in the selfgravity case we could
implement a correction in \eqref{eq:etotal2}, which accounts for a rising
"selfgravitiness" of a local fluid element like
\begin{align}
\begin{split}
\pd{t}\rho e + \pd{r_j} v_j \rho e =& -\pd{r_j}\lra{v_j p}
+\pd{r_j}\lra{v_i \sigma'_{ij}} - v_i \rho \pd{r_i} \phi\\
&+\frac{\phi_{local}}{\phi_{global}}
\lra{\frac{1}{2}\pd{t}\rho\phi-\pd{r_j}v_j\rho\phi-\rho\pd{t}\phi}
\end{split}
\end{align}
Nevertheless a more practical solution for numerical simulations might be to
refine the grid in such a way that the condition $\phi_{local} \ll
\phi_{global}$ holds everywhere in the computational domain.
\footnote{Equivalent to the criterion described in this paragraph might be the
so called Truelove criterion \citep{Truelove1997}.} 

\subsection{Divergence equation}
Using
\begin{align*}
v_j \pd{r_j}( v_i) = \pd{r_j}(v_i v_j) - v_i\pd{r_j}v_j
\end{align*}
we can express the euler equation \eqref{eq:vel} like
\begin{align*}
\pd{t}(v_i) +\pd{r_j}(v_i v_j) - v_i\pd{r_j}v_j = 
-\frac{1}{\rho}\pd{r_i}p + \frac{1}{\rho}\pd{r_j}\sigma'_{ij} + g_i.
\end{align*}
Taking the divergence of this equation yields
\begin{align*}
\pd{t}\lra{\ppd{r_i}{v_i}}&+\frac{\partial^2}{\partial r_i \partial r_j}(v_i
v_j)
-\lra{\ppd{r_i}{v_i}}\lra{\ppd{r_j}{v_j}}-v_i\pd{r_i}\lra{\ppd{r_j}{v_j}}=\\
&\frac{1}{\rho^2}\lra{\ppd{r_i}{\rho}}\lra{\ppd{r_i}{p}}
-\frac{1}{\rho}\pdd{r_i}p
-\frac{1}{\rho^2}\lra{\ppd{r_i}{\rho}}\lra{\pd{r_j}\sigma'_{ij}}
+\frac{1}{\rho}\frac{\partial^2}{\partial r_i \partial r_j}\sigma'_{ij}
+\pd{r_i}g_i
\end{align*}
With the poisson equation \eqref{eq:maxgrav1} and using the notation from
Truesdell for the divergence $\theta=\ppd{r_i}{v_i}$
we get an equation for the divergence of an arbitrary fluid
\begin{align}
\begin{split}
\pd{t}\theta-\theta^2-v_i\pd{r_i}\theta
+\frac{\partial^2}{\partial r_i \partial r_j}(v_i v_j) =&
\frac{1}{\rho^2}\lra{\ppd{r_i}{\rho}}
\lrb{\ppd{r_i}{p}-\pd{r_j}\sigma'_{ij}}\\
&-\frac{1}{\rho}\lrb{\pdd{r_i}p
-\frac{\partial^2}{\partial r_i \partial r_j}\sigma'_{ij}}
-4\pi G \rho
\end{split}
\end{align}
or in vector notation
\begin{align}
\begin{split}
\pd{t}\theta-\theta^2-(\vec{v}\cdot \nabla)\theta 
+\nabla\lrb{\nabla\cdot(v_i v_j)}=&
\frac{1}{\rho^2}\nabla \rho \cdot \lrb{\nabla p - \nabla \cdot \sigma'_{ij}}\\
&-\frac{1}{\rho}\lrb{\Delta p
-\nabla(\nabla \cdot \sigma'_{ij})}
-4\pi G \rho
\end{split}
\end{align}


\subsection{Vorticity equation}
Using the vector identity \eqref{eq:vecid02} we can express the euler 
equation \eqref{eq:vel} like
\begin{align}
\pd{t}(v_i) + \pd{r_i} \lra{\frac{1}{2} v_j v_j}
- \epsilon_{ijm} v_j \epsilon_{mkl} \pd{r_k} v_l = -\frac{1}{\rho}\pd{r_i}p +
\frac{1}{\rho}\pd{r_j}\sigma'_{ij} + g_i.
\end{align}
We can obtain an equation for the vorticity of the flow field by taking the
curl of this form of the euler equation
\begin{align}
\begin{split}
\epsilon_{ghi}\pd{r_h}\pd{t}(v_i) 
+\epsilon_{ghi}\pd{r_h} \pd{r_i}\lra{\frac{1}{2} v_j v_j}
-\epsilon_{ghi}\pd{r_h} \epsilon_{ijm} v_j \epsilon_{mkl} \pd{r_k} v_l =\\
-\epsilon_{ghi}\pd{r_h}\lra{\frac{1}{\rho}\pd{r_i}p} 
+\epsilon_{ghi}\pd{r_h}\lra{\frac{1}{\rho}\pd{r_j}\sigma'_{ij}} 
+\epsilon_{ghi}\pd{r_h} g_i.
\end{split}
\end{align}
If the time and space derivative of the velocity field commute, if
the spatial derivatives of the velocity field commute\footnote{This is true if
the velocity field is continuously differentiable twice} and if the
gravitational field can be expressed by $g_i=-\pd{r_i}\phi$ (means as the
gradient of a potential) we see that the second term on the left hand side and
also the term due to gravity are zero. This is so because these terms are
equivalent to the curl of a gradient of a scalar field which is a zero vector.
So we get
\begin{align}
\begin{split}
\pd{t}\epsilon_{ghi}\pd{r_h}v_i
-\epsilon_{ghi}\pd{r_h} \epsilon_{ijm} v_j \epsilon_{mkl} \pd{r_k} v_l =
&-\epsilon_{ghi}\lrb{\frac{1}{\rho}\pd{r_h}\lra{\pd{r_i}p} 
+ \lra{\pd{r_i}p} \lra{\pd{r_h}\frac{1}{\rho}}}\\
&+\epsilon_{ghi}\lrb{\frac{1}{\rho}\pd{r_h}\lra{\pd{r_j}\sigma'_{ij}}
+ \lra{\pd{r_j}\sigma'_{ij}} \lra{\pd{r_h}\frac{1}{\rho}}}
\end{split}
\end{align}
The first term on the right hand side is zero again, because it is the curl
of a gradient field and so we get as the vorticity equation
\footnote{An equivalent equation would be 
an equation for the rotation tensor, because the rotation tensor is 
dual to the vorticity vector (also see appendix \ref{rotstraintensor}). 
The advantage of formulating an equation for the rotation tensor would be, 
that the rotation tensor can be consistently defined in other dimensions 
than three and also in curved space.}
for some arbitrary
fluid with $\omega_g=\epsilon_{ghi}\pd{r_h}v_i$
\begin{align}
\begin{split}
\pd{t}\omega_g
-\epsilon_{ghi}\pd{r_h} \epsilon_{ijm} v_j \omega_m =
&-\frac{1}{\rho^2}\lrb{
\epsilon_{ghi} \lra{\pd{r_h}\rho} \lra{\pd{r_i}p}
-\epsilon_{ghi} \lra{\pd{r_h}\rho} \lra{\pd{r_j}\sigma'_{ij}}}\\
&+\frac{1}{\rho}\epsilon_{ghi}\pd{r_h}\pd{r_j}\sigma'_{ij}
\end{split}
\end{align}
or in vector notation with $\vec{\omega}=\nabla \times \vec{v}$
\begin{align}
\pd{t} \vec{\omega}-\nabla \times (\vec{v} \times \vec{\omega}) = 
-\frac{1}{\rho^2}\lrb{(\nabla \rho) \times(\nabla p)
- (\nabla \rho) \times (\nabla \cdot \tilde{\sigma})}
+\frac{1}{\rho} \lrb{\nabla \times (\nabla \cdot \tilde{\sigma})}
\end{align}

\subsection{Summary}
\subsubsection*{Balance equations}
\begin{align}
\pd{t}\rho + \pd{r_j}(v_j \rho) &= 0 \label{eq:mass}\\
\pd{t}(\rho v_i) + \pd{r_j}(v_j \rho v_i) &= -\pd{r_i}p + \pd{r_j}\sigma'_{ij}
+\rho g_i 
\label{eq:mom} \\
\pd{t}(\rho e) + \pd{r_j}(v_j \rho e) &= -\pd{r_j}(v_j p) + \pd{r_j}(v_i
\sigma'_{ij}) + v_i \rho g_i
\label{eq:etot}
\end{align}
with Newtonian gravity (Poisson Equation):
\begin{align}
\pd{r_j}g_j=4\pi G \rho
\end{align}
and an equation of state dependent on the material of the fluid.
\footnote{See Appendix \ref{eos}.}

\subsubsection*{Global dissipation of kinetic energy $\mathcal{E}$}
\begin{align}
\mathcal{E} = \frac{1}{V} \pd{t} E_{kin} = 
\frac{1}{V} \int_V p \pd{r_j} v_j dV 
-\frac{1}{V} \int_V \sigma'_{ij}\pd{r_j} v_i dV
-\frac{1}{V} \int_V \phi \pd{t}\rho dV
\label{eq:diss}
\end{align}

\subsubsection*{Divergence equation}
\begin{align}
\begin{split}
\pd{t}\theta-\theta^2-v_i\pd{r_i}\theta
+\frac{\partial^2}{\partial r_i \partial r_j}(v_i v_j) =&
\frac{1}{\rho^2}\lra{\ppd{r_i}{\rho}}
\lrb{\ppd{r_i}{p}-\pd{r_j}\sigma'_{ij}}\\
&-\frac{1}{\rho}\lrb{\pdd{r_i}p
-\frac{\partial^2}{\partial r_i \partial r_j}\sigma'_{ij}}
-4\pi G \rho
\end{split}
\label{eq:div}
\end{align}

\subsubsection*{Vorticity equation}
\begin{align}
\begin{split}
\pd{t}\omega_g
-\epsilon_{ghi}\pd{r_h} \epsilon_{ijm} v_j \omega_m =
&-\frac{1}{\rho^2}\lrb{
\epsilon_{ghi} \lra{\pd{r_h}\rho} \lra{\pd{r_i}p}
-\epsilon_{ghi} \lra{\pd{r_h}\rho} \lra{\pd{r_j}\sigma'_{ij}}}\\
&+\frac{1}{\rho}\epsilon_{ghi}\pd{r_h}\pd{r_j}\sigma'_{ij}
\end{split}
\label{eq:vort}
\end{align}

\section{Newtonian compressible fluid}
For a so called newtonian fluid it can be shown, that the stress tensor
$\sigma'_{ij}$ is of the form
\begin{align}
\sigma'_{ij}=2\eta S^*_{ij}+\zeta \delta_{ij}\ppd{r_k}{v_k}
\label{eq:stress}
\end{align}
with $S^*_{ij}$ being the symmetric tracefree part of the tensor
$\ppd{x_j}{v_i}$
\begin{align}
S^*_{ij}=\lrb{\frac{1}{2}\lra{\pd{r_j}v_i+\pd{r_i}v_j}-\frac{1}{3}\delta_{ij}
\ppd{r_k}{v_k}}
\end{align}
The parameter $\eta'$ is called dynamic viscosity and $\zeta$ is the so called
second dynamic viscosity. The second term of equation \eqref{eq:stress} is
often considered as small and therefore neglected. This is true in case of a
monoatomic gases there it can be shown, that $\zeta=0$ (Landau/Lifschitz 10).
In case of a incompressible fluid with constant density the term can also be
neglected, because $\ppd{r_k}{v_k} = 0$. Nevertheless for compressible fluids
(supersonic regime) $\ppd{r_k}{v_k}$ can be very large (shocks) and the second
dynamic viscosity of a n-atomic gas can not be neglected. In this case the
second term of the stress tensor additionally contributes to the pressure,
which should be considered in the equation of state. This will alter the nature
of $p$ as a thermodynamic variable, which should only depend on the local values
of $\rho$ and $e$ and not on $\ppd{r_k}{v_k}$. But since we have a stress tensor
we locally do not have a local thermodynamic equilibrium anyway, so one should
expect a change in the nature of the thermodynamic variables, which are
defined for local thermodynamic equilibrium.

\subsection{Balance equations}
In the following we will write the stress tensor for a newtonian compressible 
fluid in the form
\begin{align}
\sigma'_{ij}= \eta\lra{\pd{r_j}v_i+\pd{r_i}v_j} 
+ \lra{\zeta-\frac{2}{3}\eta}\delta_{ij}\ppd{r_k}{v_k} 
\label{eq:stress2}
\end{align}
Inserting this explicitely into the momentum equation for a compressible fluid
\eqref{eq:mom} one gets for the term
\begin{align}
\begin{split}
\pd{r_j} \sigma'_{ij} &= 
\pd{r_j}\lrb{\eta \lra{\ppd{r_j}{v_i}+\ppd{r_i}{v_j}}}
+\lra{\zeta-\frac{2}{3}\eta}\delta_{ij}\pd{r_j}\lra{\ppd{r_k}{v_k}}\\
&= \eta \lrb{\pdd{r_j}v_i +\pd{r_i}\lra{\ppd{r_j}{v_j}}}
+\lra{\zeta-\frac{2}{3}\eta}\pd{r_i}\lra{\ppd{r_k}{v_k}}\\
&=\eta \pdd{r_j}v_i
+\lra{\frac{\eta}{3}+\zeta}\pd{r_i}\lra{\ppd{r_k}{v_k}}.
\end{split}
\label{eq:divstress}
\end{align}

Inserting the stress tensor into the energy equation for a compressible fluid
\eqref{eq:etot} one gets for
\begin{align}
\begin{split}
\pd{r_j}(v_i \sigma'_{ij}) 
&=\pd{r_j}\lrb{\eta v_i \lra{\ppd{r_j}{v_i}+\ppd{r_i}{v_j}}}
+\lra{\zeta-\frac{2}{3}\eta}\pd{r_j}\lra{v_j \ppd{r_k}{v_k}}\\
&=\eta \pdd{r_j}\lra{\frac{1}{2}v_i^2}+\eta \pd{r_j}\lra{v_i\ppd{r_i}{v_j}}
+\lra{\zeta-\frac{2}{3}\eta}\pd{r_j}\lra{v_j \ppd{r_k}{v_k}}
\end{split}
\end{align}
In the end we get the following balance equations for a compressible, 
selfgravitating, newtonian fluid
\begin{align}
\pd{t}\rho + \pd{r_j}(v_j \rho) =&\ 0 \\
\pd{t}(\rho v_i) + \pd{r_j}(v_j \rho v_i) =& -\pd{r_i}p + +\rho g_i
+\eta\pdd{r_j}v_i
+\lra{\frac{\eta}{3}+\zeta}\pd{r_i}\lra{\ppd{r_k}{v_k}}\\
\begin{split}
\pd{t}(\rho e) + \pd{r_j}(v_j \rho e) =& -\pd{r_j}(v_j p) + v_i \rho g_i 
+\eta \pdd{r_j}\lra{\frac{1}{2}v_i^2}+ \eta \pd{r_j}\lra{v_i\ppd{r_i}{v_j}} \\
&+\lra{\zeta-\frac{2}{3}\eta} \pd{r_j}\lra{v_j\ppd{v_k}{r_k}}
\end{split}
\end{align}

\subsection{Global dissipation of kinetic energy $\mathcal{E}$}
Using the form \eqref{eq:stress2} we get for the term involving the stress
tensor in the equation \eqref{eq:diss} for the dissipation $\mathcal{E}$
\begin{align}
\begin{split}
\sigma'_{ij}\ppd{r_j}{v_i} 
&= \eta\lra{\pd{r_j}v_i+\pd{r_i}v_j}\ppd{r_j}{v_i}
+ \lra{\zeta-\frac{2}{3}\eta}\delta_{ij}\ppd{r_k}{v_k}\ppd{r_j}{v_i}\\
&=\frac{\eta}{2}\lra{\pd{r_j}v_i+\pd{r_i}v_j}^2
+\lra{\zeta-\frac{2}{3}\eta}\lra{\ppd{r_k}{v_k}}^2
\end{split}
\end{align}
where we also made use of the relation \eqref{eq:uscontr} for the contraction of
a symmetric with an unsymmetric tensor. With this result the global dissipation
of kinetic energy for a newtonian compressible fluid is
\begin{align}
\begin{split}
\mathcal{E} =& 
\frac{1}{V} \int_V p \ppd{r_j}{v_j} dV
-\frac{1}{V} \int_V \phi \pd{t}\rho dV
-\frac{1}{V} \int_V \frac{\eta}{2}\lra{\pd{r_j}v_i+\pd{r_i}v_j}^2 dV\\
&-\frac{1}{V} \int_V \lra{\zeta-\frac{2}{3}\eta}\lra{\ppd{r_k}{v_k}}^2 dV
\end{split}
\end{align}
This equation should be compared to equation (79,1) from \citet{Landau1991}
which additionally includes heat conduction. Nevertheless \citet{Landau1991}
seem to forget the term due to the pressure in the equation for the
dissipation.

\subsection{Divergence equation}
Using the equation for the divergence of the stress tensor
for a newtonian compressible fluid \eqref{eq:divstress} we get for
\begin{align}
\begin{split}
\pd{r_i}\lra{\pd{r_j}\sigma'_{ij}}&=
\eta\pd{r_i}\pdd{r_j}v_i+\lra{\frac{\eta}{3}+\zeta}\pdd{r_i}\theta\\
&=\eta\pdd{r_j}\theta+\lra{\frac{\eta}{3}+\zeta}\pdd{r_i}\theta\\
&=\lra{\frac{4}{3}\eta+\zeta}\pdd{r_i}\theta
\end{split}
\end{align}
By inserting this and equation \eqref{eq:divstress} into the equation for the
divergence \eqref{eq:div} we get the divergence equation for a compressible
newtonian fluid
\begin{align}
\begin{split}
\pd{t}\theta-\theta^2-v_i\pd{r_i}\theta
+\frac{\partial^2}{\partial r_i \partial r_j}(v_i v_j) =&
\frac{1}{\rho^2}\lra{\ppd{r_i}{\rho}} \cdot
\lrb{\ppd{r_i}{p}-\eta\pdd{r_j}v_i-\lra{\frac{\eta}{3}+\zeta}
\pd{r_i}\theta}\\
&-\frac{1}{\rho}\lrb{\pdd{r_i}p-\lra{\frac{4}{3}\eta+\zeta}\pdd{r_i}\theta} 
-4\pi G \rho
\end{split}
\end{align}
or in vector notation
\begin{align}
\begin{split}
\pd{t}\theta-\theta^2-(\vec{v}\cdot \nabla)\theta 
+\nabla\lrb{\nabla\cdot(v_i v_j)}=&
\frac{1}{\rho^2}\nabla \rho \cdot \lrb{\nabla p - \eta \Delta v
-\lra{\frac{\eta}{3}+\zeta}\nabla \theta}\\
&-\frac{1}{\rho}\lrb{\Delta p-\lra{\frac{4}{3}\eta+\zeta}\Delta\theta}
-4\pi G \rho
\end{split}
\end{align}
\subsection{Vorticity equation}
If we plugin the equation for the divergence of the
stress tensor for a newtonian compressible fluid \eqref{eq:divstress} into
the vorticity equation \eqref{eq:vort} we get for a compressible newtonian
fluid with $\theta=\ppd{r_k}{v_k}$
\begin{align}
\begin{split}
\pd{t}\omega_g
-\epsilon_{ghi}\pd{r_h} \epsilon_{ijm} v_j \omega_m =
&-\frac{1}{\rho^2}\epsilon_{ghi}\left[
\lra{\pd{r_h}\rho} \lra{\pd{r_i}p}
-\eta \lra{\pd{r_h}\rho} \lra{\pdd{r_j}v_i}\right.\\
&\left.-\lra{\frac{\eta}{3}+\zeta} \lra{\pd{r_h}\rho} 
\lra{\pd{r_i}\theta} \right]
+\frac{\eta}{\rho}\pdd{r_j}\omega_g \\
&+\frac{1}{\rho}\lra{\frac{\eta}{3}
+\zeta}\epsilon_{ghi}\pd{r_h}\pd{r_i}\lra{\ppd{r_k}{v_k}}
\end{split}
\end{align}
The last term on the right hand side vanishes, because it is the curl
of a gradient field. Therefore we are left with
\begin{align}
\begin{split}
\pd{t}\omega_g
-\epsilon_{ghi}\pd{r_h} \epsilon_{ijm} v_j \omega_m =
&-\frac{1}{\rho^2}\epsilon_{ghi}\left[
\lra{\pd{r_h}\rho} \lra{\pd{r_i}p}
-\eta \lra{\pd{r_h}\rho} \lra{\pdd{r_j}v_i}\right.\\
&\left.-\lra{\frac{\eta}{3}+\zeta} \lra{\pd{r_h}\rho} 
\lra{\pd{r_i}\theta} \right]
+\frac{\eta}{\rho}\pdd{r_j}\omega_g
\end{split}
\end{align}
or in vector notation with 
\begin{align}
\begin{split}
\pd{t} \vec{\omega}-\nabla \times (\vec{v} \times \vec{\omega}) = 
&-\frac{1}{\rho^2}\left[
(\nabla \rho) \times(\nabla p)
- \eta (\nabla \rho) \times (\Delta \vec{v})\right.\\
&\left.- \lra{\zeta +\frac{\eta}{3}} (\nabla \rho) \times (\nabla
\theta)\right]
+\frac{\eta}{\rho} \Delta \vec{\omega}
\end{split}
\end{align}

\subsection{Summary}
\subsubsection*{Balance equations}
\begin{align}
\pd{t}\rho + \pd{r_j}(v_j \rho) =&\ 0 \label{eq:ncmass}\\
\pd{t}(\rho v_i) + \pd{r_j}(v_j \rho v_i) =& -\pd{r_i}p + +\rho g_i
+\eta\pdd{r_j}v_i
+\lra{\frac{\eta}{3}+\zeta}\pd{r_i}\lra{\ppd{r_k}{v_k}} \label{eq:ncmom}\\
\begin{split}
\pd{t}(\rho e) + \pd{r_j}(v_j \rho e) =& -\pd{r_j}(v_j p) + v_i \rho g_i 
+\eta \pdd{r_j}\lra{\frac{1}{2}v_i^2}+ \eta \pd{r_j}\lra{v_i\ppd{r_i}{v_j}} \\
&+\lra{\zeta-\frac{2}{3}\eta} \pd{r_j}\lra{v_j\ppd{v_k}{r_k}} \label{eq:ncetot}
\end{split}
\end{align}
with Newtonian gravity (Poisson Equation):
\begin{align}
\pd{r_j}g_j=4\pi G \rho
\end{align}
and an equation of state dependent on the material of the fluid.
\footnote{See Appendix \ref{eos}.}

\subsubsection*{Global dissipation of kinetic energy $\mathcal{E}$}
\begin{align}
\begin{split}
\mathcal{E} =& 
\frac{1}{V} \int_V p \ppd{r_j}{v_j} dV
-\frac{1}{V} \int_V \phi \pd{t}\rho dV
-\frac{1}{V} \int_V \frac{\eta}{2}\lra{\pd{r_j}v_i+\pd{r_i}v_j}^2 dV\\
&-\frac{1}{V} \int_V \lra{\zeta-\frac{2}{3}\eta}\lra{\ppd{r_k}{v_k}}^2 dV
\end{split}
\label{eq:ncdiss}
\end{align}

\subsubsection*{Divergence equation}
\begin{align}
\begin{split}
\pd{t}\theta-\theta^2-v_i\pd{r_i}\theta
+\frac{\partial^2}{\partial r_i \partial r_j}(v_i v_j) =&
\frac{1}{\rho^2}\lra{\ppd{r_i}{\rho}} \cdot
\lrb{\ppd{r_i}{p}-\eta\pdd{r_j}v_i-\lra{\frac{\eta}{3}+\zeta}
\pd{r_i}\theta}\\
&-\frac{1}{\rho}\lrb{\pdd{r_i}p-\lra{\frac{4}{3}\eta+\zeta}\pdd{r_i}\theta} 
-4\pi G \rho
\end{split}
\end{align}

\subsubsection*{Vorticity equation}
\begin{align}
\begin{split}
\pd{t}\omega_g
-\epsilon_{ghi}\pd{r_h} \epsilon_{ijm} v_j \omega_m =
&-\frac{1}{\rho^2}\epsilon_{ghi}\left[
\lra{\pd{r_h}\rho} \lra{\pd{r_i}p}
-\eta \lra{\pd{r_h}\rho} \lra{\pdd{r_j}v_i}\right.\\
&\left.-\lra{\frac{\eta}{3}+\zeta} \lra{\pd{r_h}\rho} 
\lra{\pd{r_i}\theta} \right]
+\frac{\eta}{\rho}\pdd{r_j}\omega_g
\end{split}
\end{align}

\section{General incompressible fluid}
\subsection{Balance equations}
When we talk about an incompressible fluid we mean that the density of the fluid
is constant in time, i.e. $\pd{t}\rho=0$. In most cases it is also assumed that 
the fluid is not stratified, that means the density is also spatially constant,
i.e. $\pd{r_j}\rho=0$. Therefore one could call an incompressible fluid also a
constant density fluid.

With constant density the continuity equation \eqref{eq:mass} becomes
\begin{align}
\ppd{t}{\rho} + \pd{r_j}(v_j \rho) &= 0 \\
\Leftrightarrow \ppd{t}{\rho}+ v_j \ppd{r_j}{\rho} + \rho \ppd{r_j}{v_j} &= 0\\
\Rightarrow \ppd{r_j}{v_j} &= 0,\ \text{with $\rho\neq 0$} \label{eq:divzero}
\end{align}

The momentum equation \eqref{eq:mom} and the energy equation
\eqref{eq:etot} 
become
\begin{align}
\rho \lrb{\ppd{t}{v_i} + v_j \ppd{r_j}{v_i} + v_i \ppd{r_j}{v_j}}  
=& -\pd{r_i}p + +\rho g_i+ \pd{r_j}\sigma'_{ij} \\
\rho \lrb{\ppd{t}{e} + v_j \ppd{r_j}{e} + e \ppd{r_j}{v_j}} 
=& -\pd{r_j}(v_j p) + v_i \rho g_i +\pd{r_j}(v_i \sigma'_{ij})
\end{align}
If we make use of the relation \eqref{eq:divzero} in the momentum and energy 
equation we finally get as equations for an incompressible, selfgravitating, 
newtonian fluid
\begin{align}
\ppd{r_j}{v_j} =&\ 0\\
\ppd{t}{v_i} + v_j \ppd{r_j}{v_i} =& -\frac{1}{\rho}\pd{r_i}p + g_i
+\frac{1}{\rho}\pd{r_j}\sigma^*_{ij}\\
\ppd{t}{e} + v_j \ppd{r_j}{e} =& -\frac{1}{\rho} \pd{r_j}(v_j p) + v_i g_i
+\frac{1}{\rho}\pd{r_j}(v_i \sigma^*_{ij}).
\end{align}
where $\sigma^*_{ij}$ is a divergence free stress tensor.


\subsection{Divergence equation}
In the incompressible case we can set $\theta=0$ and $\pd{r_i}\rho=0$ in
equation \eqref{eq:div} and so we are left with the following equation for an
general incompressible fluid
\begin{align}
\frac{\partial^2}{\partial r_i \partial r_j}(v_i v_j) = 
-\frac{1}{\rho}\pdd{r_i}p 
+\frac{1}{\rho}\frac{\partial^2}{\partial r_i \partial r_j}\sigma^*_{ij} 
- 4\pi G \rho
\end{align}
Solving for $p$ we get
\begin{align}
\pdd{r_i}p= 
-\rho \frac{\partial^2}{\partial r_i \partial r_j}(v_i v_j)
+\frac{\partial^2}{\partial r_i \partial r_j}\sigma^*_{ij} 
-4\pi G \rho^2
\end{align}
or in vector notation
\begin{align}
\Delta p = 
-\rho \nabla\lrb{\nabla\cdot(v_i v_j)} 
+\nabla\lrb{\nabla\cdot \sigma^*_{ij}}
- 4\pi G \rho^2
\end{align}
which can be interpreted as the equation of state for a general
incompressible fluid.\footnote{Why can't we use $p=R_s \rho T$ as equation of
state for an incompressible fluid?} 

\subsection{Vorticity equation}
For an incompressible fluid $\pd{r_h}\rho = 0$ and inserting this into
the vorticity equation \eqref{eq:vort} we get
\begin{align}
\pd{t}\omega_g
&-\epsilon_{ghi}\pd{r_h} \epsilon_{ijm} v_j \omega_m =
\frac{1}{\rho}\epsilon_{ghi}\pd{r_h}\pd{r_j}\sigma^*_{ij}
\end{align}
or in vector notation
\begin{align}
\pd{t} \vec{\omega}-\nabla \times (\vec{v} \times \vec{\omega}) = 
\frac{1}{\rho} \lrb{\nabla \times (\nabla \cdot \tilde{\sigma^*})}
\end{align}

\subsection{Summary}
\subsubsection*{Balance equations}
\begin{align}
\ppd{r_j}{v_j} =&\ 0 \label{eq:icmass}\\
\ppd{t}{v_i} + v_j \ppd{r_j}{v_i} =& -\frac{1}{\rho}\pd{r_i}p + g_i
+\frac{1}{\rho}\pd{r_j}\sigma^*_{ij}\label{eq:icmom}\\
\ppd{t}{e} + v_j \ppd{r_j}{e} =& -\frac{1}{\rho} \pd{r_j}(v_j p) + v_i g_i
+\frac{1}{\rho}\pd{r_j}(v_i \sigma^*_{ij}).\label{eq:icetot}
\end{align}
with Newtonian gravity (Poisson Equation):
\begin{align}
\pd{r_j}g_j=4\pi G \rho
\end{align}
and as equation of state the "divergence equation".

\subsubsection*{Global dissipation of kinetic energy $\mathcal{E}$}
\begin{align}
\mathcal{E} = 
-\frac{1}{V} \int_V \sigma^*_{ij}\pd{r_j} v_i dV
\label{eq:icdiss}
\end{align}

\subsubsection*{Divergence equation (Equation of State)}
\begin{align}
\pdd{r_i}p= 
-\rho \frac{\partial^2}{\partial r_i \partial r_j}(v_i v_j)
+\frac{\partial^2}{\partial r_i \partial r_j}\sigma^*_{ij} 
-4\pi G \rho^2
\label{eq:icdiv}
\end{align}

\subsubsection*{Vorticity equation}
\begin{align}
\pd{t}\omega_g
&-\epsilon_{ghi}\pd{r_h} \epsilon_{ijm} v_j \omega_m =
\frac{1}{\rho}\epsilon_{ghi}\pd{r_h}\pd{r_j}\sigma^*_{ij}
\label{eq:icvort}
\end{align}


\section{Newtonian incompressible fluid}
For an incompressible newtonian fluid the terms proportional to the divergence
vanish in the stress tensor $\sigma'_{ij}$. This leads to the following
divergence free stress tensor
\begin{align}
\sigma^*_{ij}=\eta \lra{\ppd{r_j}{v_i}+\ppd{r_i}{v_j}}. 
\label{eq:nicstress}
\end{align}

\subsection{Balance equations}
Setting the divergence to zero in equation \ref{eq:divstress} we get for
the divergence of the stress tensor for an incompressible fluid
\begin{align}
\pd{r_j} \sigma^*_{ij} =\eta \pdd{r_j}v_i \label{eq:nicdivstress}
\end{align}
and for
\begin{align}
\pd{r_j}(v_i \sigma^*_{ij})
=\eta \pdd{r_j}\lra{\frac{1}{2}v_i^2}+\eta\pd{r_j}\lra{v_i\ppd{r_i}{v_j}}
\end{align}
Inserting these results into \eqref{eq:icmass}-\eqref{eq:icetot} we get
as balance equations for an incompressible newtonian fluid
\begin{align}
\ppd{r_j}{v_j} =&\ 0\\
\ppd{t}{v_i} + v_j \ppd{r_j}{v_i} =& -\frac{1}{\rho}\pd{r_i}p + g_i
+\nu\pdd{r_j}v_i\\
\ppd{t}{e} + v_j \ppd{r_j}{e} =& -\frac{1}{\rho} \pd{r_j}(v_j p) + v_i g_i
+\nu \pdd{r_j}\lra{\frac{1}{2}v_i^2}+ \nu \pd{r_j}\lra{v_i\ppd{r_i}{v_j}}
\end{align}
with the so called kinematic viscosity $\nu=\frac{\eta}{\rho}$
\footnote{Be aware that $\nu$ is a spatially independent quantity only 
for incompressible fluids! In the compressible case $\rho=\rho(x,t)$ and
therefore also the kinematic viscosity $\nu=\nu(x,t)$. So do not use $\nu$
when dealing with compressible fluid.}

\subsection{Global dissipation of kinetic energy $\mathcal{E}$}\label{nicdiss}
Inserting the incompressible newtonian stress tensor \eqref{eq:nicstress} 
in the equation for the dissipation of a general incompressible fluid
\ref{eq:icdiss} yields\footnote{The same result can be
obtained by setting $\ppd{r_j}{v_j}=0$ and $\pd{t}\rho=0$ in the equation for
the dissipation of a compressible newtonian fluid \eqref{eq:ncdiss}.}
\begin{align}
\mathcal{E} 
=-\frac{1}{V} \int_V  \eta\lra{\pd{r_j}v_i+\pd{r_i}v_j}\ppd{r_j}{v_i} dV
=-\frac{1}{V} \int_V \frac{\eta}{2}\lra{\pd{r_j}v_i+\pd{r_i}v_j}^2 dV
\end{align}
where we used again relation \ref{eq:uscontr}.

We can express this result in terms of vorticity by making use of 
equation \eqref{eq:rsrcontr} which gives us for
% \begin{align*}
% \frac{\eta}{2}\lra{\pd{r_j}v_i+\pd{r_i}v_j}^2
% &= 2 \eta S_{ij} S_{ij}
% = 2 \eta \lra{R_{ij} R_{ij} + \ppd{r_j}{v_i}\ppd{r_i}{v_j}}\\
% &= 2 \eta \lra{\frac{1}{2} \omega_k \omega_k 
% + \frac{\partial^2}{\partial r_i \partial r_j}(v_i v_j)
% - \ppd{r_i}{v_i}\ppd{r_j}{v_j}}\\
% &= \eta \omega_k \omega_k + 2 \eta \frac{\partial^2}{\partial r_i \partial
% r_j}(v_i v_j)
% \end{align*}
\begin{align*}
\frac{\eta}{2}\lra{\pd{r_j}v_i+\pd{r_i}v_j}^2
&= 2 \eta S_{ij} S_{ij}
= 2 \eta \lra{R_{ij} R_{ij} + \ppd{r_j}{v_i}\ppd{r_i}{v_j}}\\
&= 2 \eta \lra{\frac{1}{2} \omega_k \omega_k 
+ \pd{r_j}\lra{v_i \ppd{r_i}{v_j}} 
- v_j \frac{\partial^2}{\partial r_i \partial r_j} v_i}\\
&= 2 \eta \lra{\frac{1}{2} \omega_k \omega_k 
+ \pd{r_j}\lra{v_i \ppd{r_i}{v_j}} 
- v_j \frac{\partial^2}{\partial r_j \partial r_i} v_i}\\ 
 &= \eta \omega_k \omega_k + 2 \eta \pd{r_j}\lra{v_i \ppd{r_i}{v_j}}
\end{align*}
So we can express the dissipation for an incompressible newtonian fluid as
\begin{align}
\mathcal{E} = 
-\frac{1}{V} \int_V \eta \omega_k \omega_k dV
-\frac{1}{V} \int_V 
2 \eta \pd{r_j}\lra{v_i \ppd{r_i}{v_j}} dV \label{eq:nicdissfull}
\end{align}
Using Gauss' theorem to transform the second term we can express this like
\begin{align*}
\mathcal{E} = 
-\frac{1}{V} \int_V \eta \omega_k \omega_k dV
-\frac{1}{V} \oint_A 
2 \eta v_i \ppd{r_i}{v_j} dA
\end{align*}
So if $v_i \ppd{r_i}{v_j}=0$ on the border of the volume $V$ the second term 
will vanish
and the dissipation in a incompressible, newtonian fluid is 
due to the first term only \footnote{Have a look at the Appendix
\ref{contradiss} for an simple example of a flow field, where the second term 
doesn't
vanish!}
\footnote{Very often $\rho$ is moved to the other side of the equation (which is 
possible, because we assume it is spatially constant) and so we get 
$\epsilon = \frac{1}{V}\pd{t}\int_V v^2 dV = -\frac{\nu}{V} \int_V \omega^2 dV$}
 \begin{align}
\mathcal{E} = -\frac{\eta}{V} \int_V \omega^2 dV \label{eq:nicdissvort}
\end{align}
From looking at the divergence equation (see below) we can derive another 
relation, which tells us, when the second term in equation 
\eqref{eq:nicdissfull} will vanish
\begin{align*}
\frac{1}{V} \int_V 2\eta \pd{r_j}\lra{v_i \ppd{r_i}{v_j}} dV 
&=\frac{1}{V} \int_V 2\eta \frac{\partial^2}{\partial r_i \partial r_j}(v_i v_j)dV\\
&= \frac{1}{V} \int_V \frac {2\eta}{\rho}\lra{\pdd{r_i}p  + 4\pi G \rho^2} dV
=  \frac{2 \nu}{V} \int_V \lra{\pdd{r_i}p  + \rho \pdd{r_i}\phi} dV\\
&= \frac{2 \nu}{V} \oint_A \pd{r_i}p dA + \frac{2 \eta}{V} \oint_A \pd{r_i}\phi dV
\end{align*}
where we made use of the poisson equation for the gravitational potential and 
Gauss' theorem. So only if the pressure gradient and the gravitational force 
balance at the surface of the fluid we can neglect the second term \footnote{In 
case of no gravity the pressure gradient has to be zero at the surface.}. 

We hope that from this discussion the assumptions behind equation 
\eqref{eq:nicdissvort}
become clear compared to the rather obscure arguments by \citet{Frisch1995}. 
Nevertheless it is still unknown weather our reasoning and the reasoning of 
\citet{Frisch1995} is 
equivalent.

\subsection{Divergence equation}
From equation \eqref{eq:nicdivstress} we get for
\begin{align}
\frac{\partial^2}{\partial r_i \partial r_j}\sigma^*_{ij} 
= \eta \pd{r_i} \lra{\pdd{r_j}v_i} 
= \eta \pdd{r_j} \lra{\ppd{r_i}{v_i}} 
= 0
\end{align}
Using this result in equation \eqref{eq:icdiv} we get as equation of state for
an incompressible newtonian fluid
\begin{align}
\pdd{r_i}p= 
-\rho \frac{\partial^2}{\partial r_i \partial r_j}(v_i v_j)
-4\pi G \rho^2
\end{align}
or in vector notation
\begin{align}
\Delta p = 
-\rho \nabla\lrb{\nabla\cdot(v_i v_j)}
- 4\pi G \rho^2
\end{align}
Actually the divergence equation for an incompressible newtonian fluid 
has a very interesting form. It is show in the Appendix \ref{diveq} that the 
divergence equation might be related to Bernoulli's law and even to the 
Einstein equation of general relativity.  
 
\subsection{Vorticity equation}
Inserting equation \eqref{eq:nicdivstress} into \ref{eq:icvort}
we get the vorticity equation for an incompressible newtonian fluid
\begin{align}
\pd{t}\omega_g
-\epsilon_{ghi}\pd{r_h} \epsilon_{ijm} v_j \omega_m =
\frac{\eta}{\rho}\pdd{r_j}\omega_g
\end{align}
or in vector notation
\begin{align}
\pd{t} \vec{\omega}-\nabla \times (\vec{v} \times \vec{\omega}) = 
\frac{\eta}{\rho} \Delta \vec{\omega}.
\end{align}

\subsection{Summary}

\subsubsection*{Balance equations}
\begin{align}
\ppd{r_j}{v_j} =&\ 0\\
\ppd{t}{v_i} + v_j \ppd{r_j}{v_i} =& -\frac{1}{\rho}\pd{r_i}p + g_i
+\nu\pdd{r_j}v_i\\
\ppd{t}{e} + v_j \ppd{r_j}{e} =& -\frac{1}{\rho} \pd{r_j}(v_j p) + v_i g_i
+\nu \pdd{r_j}\lra{\frac{1}{2}v_i^2}+ \nu \pd{r_j}\lra{v_i\ppd{r_i}{v_j}}
\end{align}
with Newtonian gravity (Poisson Equation):
\begin{align}
\pd{r_j}g_j=4\pi G \rho
\end{align}
and as equation of state the "divergence equation".

\subsubsection*{Global dissipation of kinetic energy $\mathcal{E}$ 
\footnote{Have a look at section \ref{nicdiss} to understand the assumptions
made when deriving this equation.}}
\begin{align}
\mathcal{E} = -\frac{\eta}{V} \int_V  \omega^2 dV
\label{eq:nicdiss}
\end{align}

\subsubsection*{Divergence equation (Equation of State)}
\begin{align}
\pdd{r_i}p= 
-\rho \frac{\partial^2}{\partial r_i \partial r_j}(v_i v_j)
-4\pi G \rho^2
\end{align}

\subsubsection*{Vorticity equation}
\begin{align}
\pd{t}\omega_g
-\epsilon_{ghi}\pd{r_h} \epsilon_{ijm} v_j \omega_m =
\frac{\eta}{\rho}\pdd{r_j}\omega_g
\end{align}

\section{Fluid dynamics in comoving coordinates}
\subsection{Introduction}
On large scales ($>$ 100Mpc) the distribution of matter 
in the universe is isotropic (it looks the same in all directions) 
and homogeneous (it is isotropic at each point). But only the space 
is assumed to be isotropic and homogenous. The observed expansion of 
the universe singles out a special direction in time.\footnote{The 
universe is not a maximally symmetric 4-dimensional manifold, but can be 
depicted as maximally symmetric 3-dimensional sheets spacelike sheets 
in 4-dimensional spacetime. The metric on such a manifold is the 
Robertson-Walker-metric.}

The physical distance on large scales\footnote{This is a very important 
point. If the space would also
expand on small scales we couldn't measure the expansion, because everything 
including our distance measurement device would expand. But on small scales
the universe is not homogenous. On small scales the metric of the universe is 
not a Robertson-Walker metric, but more like a Schwarzschild metric, which
is isotropic, but not homogenous.} 
between two points in such an expanding universe varies with time like
\begin{align}
r_i=a(t) x_i
\end{align}
The factor $a$ is a dimensionless scale factor greater than zero, which must 
be the same for each component of the distance vector because of the assumed isotropy.
The scale factor can only depend on the time $t$ and not on the position
$x_i$ because of the assumed homogenity of space. 

The change of the distance with time in an expanding universe is then
\begin{align}
\dot{r}_i = \dot{a} x_i + a \dot{x}_i  
\end{align}
The global velocity of a particle $v_i = \dot{r}_i$ which does not move 
relative to the expanding space ($\dot{x}_i = 0$) is then
\begin{align}
\dot{v}_i = \dot{a} x_i = \fh r_i = H(t) r_i 
\end{align}
where $H$ is the so called Hubble parameter. Is a particle moving relative to
the expanding space ($\dot{x}_i \neq 0$) then we measure the additional 
local (also called proper) velocity $u_i = a \dot{x}_i$ of the particle. This local
velocity can, according to special relativity, be never greater than the speed of light
$c$. Nevertheless the global velocity (e.g. the measured escape velocities of galaxies at 
great distances) can be greater than $c$ \citep{Davis2004}. Generally the physical velocity 
of a particle is the the sum of global and local velocity
\begin{align}
v_i = \dot{a} x_i + u_i (x_i,t)
\end{align}



\subsubsection{Useful transformations}
\begin{align}
r_i&=a(t) x_i\\
v_i&= a \dot{x}_i + \dot{a} x_i = u_i + \dot{a} x_i\\
\pd{r_i}&=\fa\pd{x_i}\\
\pd{r_i}v_i&=\fa\pd{x_i}u_i+3\fh\label{eq:cotrans4}\\
\pd{r_i}v_j&=\fa\pd{x_i}u_j + \fh\delta_{ij} \label{eq:cotrans5}\\
\lra{\frac{\partial A}{\partial t}}_r+v_j \frac{\partial A}{\partial r_j} &= 
\lra{\frac{\partial A}{\partial t}}_x+\fa u_j \frac{\partial A}{\partial
x_j}\\
A(r_i,v_i,t) &\neq A(x_i,u_i,t)
\end{align}
The stress tensor for a newtonian fluid in comoving coordinates is\footnote{See
Appendix \ref{costress}.}
\begin{align}
\sigma'_{ij}=2\eta T^*_{ij}+\zeta\delta_{ij}\fa\lra{\pd{x_k}u_k+n\dot{a}}
\end{align}
with 
\begin{align}
T^*_{ij}=\fa\lrb{\frac{1}{2}\lra{\pd{x_j}u_i+\pd{x_i}u_j}
-\frac{1}{n}\delta_{ij}\pd{x_k}u_k}
\end{align}

\subsubsection{Transformed equations}
\begin{align}
\pd{t}\rho + \fa\pd{x_j}(u_j \rho) =& -3\fh\rho  \\
\pd{t}(\rho u_i) + \fa\pd{x_j}(u_j \rho u_i) =& 
-\fa\pd{x_i}p + \fa\pd{x_j}\sigma'_{ij} +\rho g^*_i -4\fh \rho u_i
\label{eq:commom}
\\
\begin{split}
\pd{t}(\rho e) + \fa\pd{x_j}(u_j \rho e) =& 
-\fa\pd{x_j}(u_j p) + \fa\pd{x_j}(u_i \sigma'_{ij}) + \fa u_i \rho g^*_i \\ 
&- 3 \fh(\rho e +\frac{1}{3}\rho u^2_i+p)\label{eq:cometot}
\end{split}
\end{align}
with
\begin{itemize}
\item Newtonian Gravity in comoving coordinates (Poisson Equation):
 
$\fa\pd{x_j}g^*_j=4\pi G$
with
$g_i^*=\fa\frac{\partial \varphi}{\partial x_i}$ and
$\varphi=\phi+\frac{1}{2}a\ddot{a}x_i^2$
\end{itemize}
The energy equation is the sum of the equation for the kinetic energy and the
internal energy
\begin{align}
\pd{t}(\rho e_k)+\fa\pd{x_j}(u_j \rho e_k) &= -\fa u_i \pd{x_i}p
+\fa u_i \pd{x_j}\sigma'_{ij}-\fa\rho u_i g^*_i-5\fh\rho e_k\\
\pd{t}(\rho e_{int})+\fa\pd{x_j}(u_j \rho e_{int})&=
-\fa p \pd{x_j}u_j -\fa \sigma'_{ij}\pd{x_j}u_i- 3\fh(\rho e_{int} +p)
\end{align}

\newpage
\appendix
\section{Equations of State} \label{eos}
\subsection{Ideal gas}
The ideal gas law (the thermic equation of state for an ideal gas)
is
\begin{align}
p V = N k_B T, \label{eq:igl}
\end{align}
where $N$ is the number of molecules of the gas, $k_B$ is the Boltzmann
constant and $p$, $V$, $T$ are the state variables pressure, volume and
temperature respectivly. We can rewrite this equation also in the following 
ways
\begin{align*}
p V &= \frac{N}{N_A} N_A k_B  T \\ 
	 &= n R T \\
    &= \frac{m}{M} R T \\
\Rightarrow p &= \frac{m}{V} \frac{R}{M} T \\
              &= \rho R_S T
\end{align*}
where we used the number of moles $n=\frac{N}{N_A}$, 
the molar mass of a gas molecule $M = \frac{m}{n}$ ($m$ is the
mass of the gas molecule), the universal gas constant $R= N_A k_B$ and the
specific gas constant $R_S=\frac{R}{M}$.

From statistical thermodynamics it follows that the
internal energy $U$ of an ideal gas is related to the temperature via the
so called caloric equation of state
\begin{align}
U = \frac{f}{2}N k_B T. \label{eq:caleos}
\end{align}
$f$ specifies the number of degrees of freedom of a molecule (3 for a 
one-atomic gas, 5 for a two-atomic gas and 6 for a nonlinear many-atomic
gas)\footnote{These values are for a three dimensional world! If we would be
in a two-dimensional world we would have 2 for a 
one-atomic gas, 3 for a two-atomic gas and 3 for
any asymmetric many-atomic gas.}.
The adiabatic coefficient $\gamma$ is related to $f$ via
\begin{align}
\gamma = \frac{c_p}{c_v} = 1 + \frac{2}{f} \label{eq:gam}
\end{align}
where $\frac{c_p}{c_v}$ is the ratio of the specific
heat capacity at constant pressure $c_p$ to the specific heat capacity at
constant volume $c_v$ of an ideal gas.\footnote{Is this relation dependent on
the number of dimensions we live in?}

Inserting \eqref{eq:igl} and \eqref{eq:gam} into \eqref{eq:caleos} we get for
the pressure of an ideal gas
\begin{align}
p=(\gamma-1) \rho u \label{eq:igpress}
\end{align}
where $u$ is the specific internal energy. The specific internal energy can be
computed from the total energy \footnote{This depends
 on the definition of the total energy! If one includes the potential energy
in the total energy one has to substract also the potential energy to get the
internal energy.} via $u=e-\frac{1}{2}v^2$ and so we get for the
pressure
\begin{align}
p=\lra{\gamma-1} \rho \lra{e-\frac{1}{2}v^2}
\end{align}

\subsection{Van-der-Waals-Gas}

...to be completed

\subsection{Relativistic Gas}

...to be completed


\section{The speed of sound}
The speed of sound is defined as the speed of propagation of small pertubations
in density and pressure in a fluid in a compressible fluid. For the derivation
of the sound speed it is also assumed that the fluid is homogeneous, i.e. the
average value of density $\rho_0$ and pressure $p_0$ are spatially (and
temporarily ?) constant. Then we can split the local value of density and
pressure in a spatially constant average value and a small spatially an
temporarily varying pertubation $\rho'$ and $p'$ respectively
\begin{align}
\rho &= \rho_0 + \rho'(x,t),\ \rho' \ll \rho_0 \\
p &= p_0 + p'(x,t),\ p' \ll p_0
\end{align}
For an ideal fluid a temporal change in density or pressure has to adiabatic,
i.e. with constant entropy. The derivation of the equation of motion for a small
pertubation in density and pressure then leads us to a wave equation with a
propagation speed
\begin{align}
c=\sqrt{\lra{\frac{\partial p_0}{\partial \rho_0}}_S}
\end{align}
which is called the adiabatic speed of sound. 

Further on for an adiabatic change of state it applies
\begin{align}
p V^\gamma &= \text{const.} = C \ \ |\ \cdot\frac{1}{M^\gamma}\\
p \lra{\frac{V}{M}}^\gamma &= \frac{C}{M^\gamma}\\
p \lra{\frac{1}{\rho}}^\gamma &= C^*\\
\Longrightarrow 
p &= C^* \rho^\gamma \Longleftrightarrow \frac{p}{\rho} = C^* \rho^{\gamma-1}
\end{align}
Using this we get
\begin{align}
\frac{\partial p_0}{\partial \rho_0} = \gamma C^* \rho_0^{\gamma-1} 
= \gamma \frac{p_0}{\rho_0}.
\end{align}
With this result the adiabatic speed of sound for an ideal, compressible,
homogeneous fluid yields
\begin{align}
c=\sqrt{\gamma \frac{p_0}{\rho_0}}
\end{align}
Using equation \eqref{eq:igpress} we can compute the sound speed for an ideal gas 
directly from the internal energy
\begin{align}
c=\sqrt{\gamma (\gamma-1) u_0}
\end{align}
\subsection{Machnumber}
The Machnumber of a flow is defined as
\begin{align}
\text{Ma}=\frac{v}{c}
\end{align}
The Machnumber can also be interpreted as the ratio of the kinetic and internal 
energy, because
\begin{align}
\text{Ma}^2=\frac{v^2}{\gamma(\gamma-1) u}=\frac{1}{\gamma(\gamma-1)}\frac{e_{kin}}{e_{int}}
\end{align}
So, the Machnumber $Ma = x \cdot \sqrt{e_{kin}/e_{int}}$ where $x$ for some 
values of $\gamma$ can be found in the following table:  
\begin{center}
\begin{tabular}{ccc}
$\gamma$ & $\gamma(\gamma-1)$ & $x = \lrb{\gamma(\gamma-1)}^{-1/2}$ \\ 
\hline
\hline
$\frac{5}{3}$ & $\frac{10}{9}$ & $\frac{3}{\sqrt{10}}\approx 0.949$ \\  
$\frac{7}{5}$ & $\frac{14}{25}$ & $\frac{5}{\sqrt{14}}\approx 1.336$ \\  
$\frac{4}{3}$ & $\frac{4}{9}$ &  $\frac{3}{2}$ = 1.5\\ 
\hline 
\end{tabular}
\end{center} 

\section{Dimensional analysis}
We want to write down the general equations of fluid dynamics in dimensionless
form. Therefore we introduce the following dimensionless quantities
\begin{align*}
r_i^*=\frac{r_i}{l_0} &\Rightarrow
\pd{r_i}=\pd{r_i^*}\ppd{r_i}{r_i^*}=\frac{1}{l_0}\pd{r_i^*}\\
v_i^*=\frac{v_i}{v_0}\\
t^*=\frac{t}{l_0} &\Rightarrow
\pd{t}=\pd{t^*}\ppd{t}{t^*}=\frac{1}{t_0}\pd{t^*}\\
\rho^*=\frac{\rho}{\rho_0}
\end{align*}
Inserting these into the continuity equation \eqref{eq:mass} we get
\begin{align*}
\frac{\rho_0}{t_0}\pd{t^*}\rho^* 
+ \frac{v_0 \rho_0}{l_0} \pd{r_j^*}(v_j^*\rho^*) &= 0 \ | 
\cdot \frac{l_0}{\rho_0 v_0}\\
\underbrace{\frac{l_0}{v_0 t_0}}_{Sr}\pd{t^*}\rho^* 
+ \pd{r_j^*}(v_j^*\rho^*) = 0
\end{align*}
This derivation shows, that solutions of the continuity equation
are similar, if the Strouhal number $Sr = \frac{l_0}{v_0 t_0}$ is the
same. Flows with a Strouhal number $Sr=0$, are so called stationary flows.
Nevertheless the Strouhal number is most often set to one, by assuming
$v_o=\frac{l_0}{t_0}$. 
Using the additional dimensionless quantities
$p^*=\frac{p}{p_0},\sigma_{ij}^*=\frac{\sigma_{ij}}{\sigma_0},
g^*=\frac{g}{g_0 }$
in the momentum equation \eqref{eq:mom} yields
\begin{align*}
\frac{\rho_0 v_0}{t_0}\pd{t^*}(\rho^* v_i^*) 
+ \frac{\rho_0 v_0^2}{l_0}\pd{r_j^*}(v_j^* \rho^* v_i^*) &= 
- \frac{p_0}{l_0}\pd{r_i^*}p^* 
+\frac{\sigma_0}{l_0}\pd{r_j^*}\sigma_{ij}^*
+\rho_0 g_0 \rho^* g_i^*\ | \cdot \frac{l_0}{\rho_0 v_0^2}\\
\underbrace{\frac{l_0}{v_0 t_0}}_{Sr}\pd{t^*}(\rho^* v_i^*) 
+ \pd{r_j^*}(v_j^* \rho^* v_i^*) &= 
-\underbrace{\frac{p_0}{\rho_0 v_0^2}}_{Ma^{-2}_{iso}}\pd{r_i^*}p^* 
+\underbrace{\frac{\sigma_0}{\rho_0 v_0^2}}_{Re^{-1}}\pd{r_j^*}\sigma_{ij}^*
+\underbrace{\frac{\rho_0 g_0 l_0}{\rho_0 v_0^2}}_{Fr^{-1}} \rho^* g_i^*
\end{align*}
The occuring dimesionless numbers are the isothermal Mach number $Ma_{iso}$,
which is related to the Euler number $Eu$ or the Ruark number $Ru$ like
$Ma^2_{iso}=Eu=Ru^{-1}$, the Froude number $Fr$, which is related to the
Richardson number $Ri$ like $Fr=Ri^{-1}$ and the Reynolds number $Re$.
All these numbers measure the importance of the term they are related to
compared to the nonlinear advection term $\pd{r_j^*}(v_j^* \rho^* v_i^*)$,
e.g. for high Mach numbers the pressure term $\pd{r_i^*}p^*$ becomes less and
less important compared to the advection term; for high Reynolds numbers the 
stress term $\pd{r_j^*}\sigma_{ij}^*$ becomes less and less important compared
to the advection term and the equation shows more and more nonlinear behaviour.
For a newtonian fluid with
\begin{align}
\sigma_{ij} \simeq \eta \ppd{r_j}{v_i} = 
\frac{\eta_0 v_0}{l_0} \eta^* \ppd{r_j^*}{v_i^*},
\end{align}
where we introduced the dimensionless quantity $\eta^*=\frac{\eta}{\eta_0}$,
we get $\sigma_0= \frac{\eta_0 v_0}{l_0}$ and therefore we can express the
Reynolds number\footnote{We neglected the second viscosity $\zeta$. In 
principle there exists a second Reynolds number $Re_2=\frac{\rho_0 l_0 v_0
}{\zeta_0}$} like
\begin{align}
Re=\frac{l_0 \rho_0 v_0^2 }{\eta_0 v_0} = \frac{\rho_0 l_0 v_0 }{\eta_0}
\end{align}
Playing the same game with the equation for the internal energy \ref{eq:eint}
using $e_{int}^*=\frac{e_{int}}{u_0}$ we get
\begin{align*}
\underbrace{\frac{l_0}{v_0 t_0}}_{Sr} \pd{t^*}\rho^* e_{int}^* 
+ \pd{r_j^*} v_j^* \rho^* e_{int}^* &= 
-\underbrace{\frac{p_0}{\rho_0 u_0}}_{Ga_1} p^*\pd{r_j^*}v_j^*
+\underbrace{\frac{\sigma_0}{\rho_0 u_0}}_{Ga_2} \sigma_{ij}^* \pd{r_j^*}v_i^*
\end{align*}
The new dimensionles quantities that occur in the energy equation seem to have
no name in the literature, but we will call them "Gamma1" (Ga1) and "Gamma2"
(Ga2) for now, since they are related to the adiabatic coefficient.
This can be seen by replacing $p_0 p^*$ according to equation
\begin{align}
p_0 p^* = (\gamma-1) \rho_0 u_0 \rho^* e_{int}^*
\end{align}
which is valid for an an ideal, nonisothermal ($\gamma \neq 1$) gas. Doing this
we get\footnote{We cannot get rid of Ga2 in the same way, since therefore we
would have to assume an equation relating $\sigma_0 \sigma_{ij}^*$ to the
internal energy. But this would only be possible, if we would assume that the
internal energy is a tensorial quantity, which is not the way how
internal energy is defined normally.}
\begin{align*}
 Sr \cdot \pd{t^*}\rho^* e_{int}^* 
+ \pd{r_j^*} v_j^* \rho^* e_{int}^* &= 
-(\gamma-1) \rho^* e_{int}^* \pd{r_j^*}v_j^*
+Ga_2 \cdot \sigma_{ij}^* \pd{r_j^*}v_i^*
\end{align*}
For a selfgravitating fluid we even have on more dimensionless quantity, which
appears, when we write down the dimesionless form of the poisson equation of
gravity
\begin{align}
\underbrace{\frac{g_0}{4\pi G \rho_0 l_0}}_{C_G}\ppd{r_i^*}{g_i^*} = \rho^*.
\end{align}
But this quantity $C_G$ also seems to have no name in the literature
\citep[e.g.][]{Durst2007}.

\section{Color fields}
There are two possibilities to implement a color field $c$ in a fluid code. One
can treat color like the density, i.e. the color variable obeys a conservation
law like the density
\begin{align}
\pd{t} c + \pd{r_j}(v_j c) &= 0.
\end{align}
In this case $c$ will exactly behave like density if density and color have the 
same initial value.

On the other hand one can treat it like a specific quantity obeying a
conservation law like
\begin{align}
\pd{t} \rho c + \pd{r_j}(v_j \rho c) &= 0.
\end{align}
In this case we see, that if $c$ is spatially constant at a time $t_0$, i.e. 
$c(t_0)=c_0$, $\ppd{r_j}{c(t_0)}=0$, it will stay constant forever
\begin{align}
\pd{t} \rho c + \pd{r_j}(v_j \rho c) &= 0 \\
\Leftrightarrow \ppd{t}{c} + v_j \ppd{r_j}{c} &= 0 \\
\Leftrightarrow \td{t} c &=0\\
\Rightarrow c &= const. = c_0
\end{align}

\section{Fluid dynamics in one dimension}
\subsection{Balance equations in Eulerian form}
In one dimension all vector and tensor quantities from the balance equations
\eqref{eq:mass}-\eqref{eq:etot} degenerate to scalar quantities, eg.
\begin{align}
v_i,v_j &\longrightarrow v_1=v\\
\sigma'_{ij} &\longrightarrow \sigma'_{11}=\sigma'\\
g_i &\longrightarrow g_1=g
\end{align}
The balance equations of fluid dynamics in one dimensions are therefore written
like:
\begin{align}
\pd{t}\rho + \pd{r}(v \rho) &= 0 \label{eq:1dmass}\\
\pd{t}(\rho v) + \pd{r}(v \rho v) &= -\pd{r}p + \pd{r}\sigma' -\rho g
\label{eq:1dmom} \\
\pd{t}(\rho e) + \pd{r}(v \rho e) &= -\pd{r}(v p) + \pd{r}(v \sigma') - v \rho g
\label{eq:1detot}
\end{align}
\subsection{Balance equations in Lagrangian form}
By using the Euler derivate we can write \eqref{eq:1dmass}-\eqref{eq:1detot}
like
\begin{align}
\td{t}\rho + \rho \pd{r} v &= 0 \\
\td{t}(\rho v) + \rho v \pd{r} v &= -\pd{r}p + \pd{r}\sigma' -\rho g  \\
\td{t}(\rho e) + \rho e \pd{r} v &= -\pd{r}(v p) + \pd{r}(v \sigma') - v \rho g
\end{align}
If we introduce the so called mass coordinate defined by
\begin{align}
\partial m = \rho\cdot\partial r \Rightarrow \partial r = \frac{\partial
m}{\rho}
\end{align}
we can write \eqref{eq:1dmass} like
\begin{align}
&&\td{t}\rho + \rho^2 \pd{m} v &= 0 \\
\Leftrightarrow&& -\frac{1}{\rho^2}\td{t}\rho - \pd{m} v &= 0 \\
\Leftrightarrow&& \td{t}\lra{\frac{1}{\rho}} &= \pd{m} v
\end{align}
The momentum equation \eqref{eq:1dmom} is transformed like
\begin{align}
&& \td{t}(\rho v) + \rho^2 v \pd{m} v &= -\rho \pd{m}p + \rho \pd{m}\sigma'
-\rho g\\
\Leftrightarrow&& \frac{1}{\rho}\lra{\rho \td{t}v + v\td{t}\rho} 
+ \rho v \pd{m}v &= -\pd{m}p + \pd{m}\sigma' - g \\
\Leftrightarrow&& \td{t}v + v \lra{\frac{1}{\rho}\td{t}\rho+\rho \pd{m} v} &=
-\pd{m}p + \pd{m}\sigma' - g \\
\Leftrightarrow&& \td{t}v - \rho v
\underbrace{\lra{-\frac{1}{\rho^2}\td{t}\rho-\pd{m} v}}_{0} &=
-\pd{m}p + \pd{m}\sigma' - g\\
\Leftrightarrow&& \td{t}v &= -\pd{m}p + \pd{m}\sigma' - g
\end{align}
With an analogous transformation the energy equation \eqref{eq:1detot} can be
written like
\begin{align}
\td{t}e &= -\pd{m}(vp) + \pd{m}(v\sigma') - vg
\end{align}
Summing up the three equations an retransforming them in terms of $dr$ we get
the balance equations in one-dimensional lagrangian form as they are often used 
in numerical fluid dynamics:
\begin{align}
\td{t}\lra{\frac{1}{\rho}} &= \frac{1}{\rho}\pd{r} v \\
\td{t}v &= -\frac{1}{\rho}\pd{r}p + \frac{1}{\rho}\pd{r}\sigma' - g\\
\td{t}e &= -\frac{1}{\rho}\pd{r}(vp) + \frac{1}{\rho}\pd{r}(v\sigma') - vg
\end{align}




\section{Rate of strain tensor, rotation tensor}\label{rotstraintensor}
The decomposition of the jacobian of the velocity field into a symmetric and
an antisymmetric part yields
\begin{align}
\ppd{r_j}{v_i}&=
\underbrace{\frac{1}{2}\lra{\ppd{r_j}{v_i}+\ppd{r_i}{v_j}}}_{S_{ij}=S_{ji}}
+\underbrace{\frac{1}{2}\lra{\ppd{r_j}{v_i}-\ppd{r_i}{v_j}}}_{R_{ij}=-R_{ji}}
\end{align}
The symmetric part $S_{ij}$ is called \emph{rate of strain tensor} and the
antisymmetric part is called \emph{rotation tensor}. $R_{ij}$ has only three
independent components and can be expressed in terms of a (pseudo-)
3-vector.\footnote{$S_{ij}$ has six independent components. Is it possible to
find a representation in terms of two (pseudo-) 3-vectors?}
This vector is equivalent to the negative curl of the velocity field 
$\vec{\omega}=\nabla \times \vec{v}$ as we can see by multiplying $R_{jk}$ with
$\epsilon_{ijk}$
\begin{align*}
\frac{1}{2}\epsilon_{ijk}R_{jk}&=\frac{1}{2}\epsilon_{ijk}\lra{\ppd{r_k}{v_j}
-\ppd { r_j } { v_k}}=
\frac{1}{2}\epsilon_{ijk}\ppd{r_k}{v_j}-\frac{1}{2}\epsilon_{ijk}\ppd{r_j}{v_k}
\\
&=
\frac{1}{2}\epsilon_{ijk}\ppd{r_k}{v_j}-\frac{1}{2}\epsilon_{ikj}\ppd{r_k}{v_j}
=
\frac{1}{2}\epsilon_{ijk}\ppd{r_k}{v_j}+\frac{1}{2}\epsilon_{ijk}\ppd{r_k}{v_j}
\\
&=
\epsilon_{ijk}\ppd{r_k}{v_j}
=\epsilon_{ikj}\ppd{r_j}{v_k}
=-\epsilon_{ijk}\ppd{r_j}{v_k}\\
&=-\omega_i
\end{align*}
Further one can show that
\begin{align*}
-\frac{1}{2}\epsilon_{ijk}\omega_k
&=-\frac{1}{2}\epsilon_{ijk}\epsilon_{klm}\ppd{r_l}{v_m}
=-\frac{1}{2}\epsilon_{kij}\epsilon_{klm}\ppd{r_l}{v_m}\\
&=-\frac{1}{2}\lra{\delta_{il}\delta_{jm}-\delta_{im}\delta_{jl}}\ppd{r_l}{v_m}
\\
&=-\frac{1}{2}\lra{\ppd{r_i}{v_j}-\ppd{r_j}{v_i}}
=\frac{1}{2}\lra{\ppd{r_j}{v_i}-\ppd{r_i}{v_j}}\\
&=R_{ij}
\end{align*}
and
\begin{align}
R_{ij} R_{ij} 
&=-\frac{1}{2}\epsilon_{ijk}\omega_k \cdot -\frac{1}{2}\epsilon_{ijl}\omega_l\\
&=\frac{1}{4}\epsilon_{ijk}\epsilon_{ijl}\omega_k \omega_l
=\frac{1}{4} \cdot 2 \delta_{kl} \omega_k \omega_l\\
&=\frac{1}{2} \omega_k \omega_k
\label{eq:rrcontr}
\end{align}
Additionally the following relation between the contraction of rate of strain
tensor and the contraction of the rotation tensor is useful
\begin{align}
\begin{split}
4 S_{ij}S_{ij} -  4 R_{ij}R_{ij} &=
\lra{\ppd{r_j}{v_i}+\ppd{r_i}{v_j}}^2 - \lra{\ppd{r_j}{v_i}-\ppd{r_i}{v_j}}^2\\ 
&=\lra{\ppd{r_j}{v_i}}^2 + \lra{\ppd{r_i}{v_j}}^2 +
2\ppd{r_j}{v_i}\ppd{r_i}{v_j}
-\lra{\ppd{r_j}{v_i}}^2 - \lra{\ppd{r_i}{v_j}}^2 +
2\ppd{r_j}{v_i}\ppd{r_i}{v_j}\\ 
&= 4 \ppd{r_j}{v_i}\ppd{r_i}{v_j}
\end{split}
\label{eq:rsrcontr}
\end{align}
\section{Derivation of the stress tensor for a newtonian fluid}
It is generally assumed, that friction between fluid elements is proportional
to the area of their surfaces. So in general the frictional or vicous force
on a fluid element can be expressed like
\begin{align}
F_{visc,i}= \oint_A \sigma'_{ij} n_j dA = \int_V \pd{r_j} \sigma'_{ij} dV.
\end{align}
This force leads to an irreversible rise of temperature in the fluid or an irreversible
decrease of kinetic energy expressed by the equation for the dissipation
\begin{align}
\pd{t} E_{kin,visc} = - \int_V \sigma'_{ij} \pd{r_j} v_i dV
\end{align}
For a motionless fluid ($v_i=0$) and for a fluid with constant velocity 
($\ppd{r_j}{v_i} = 0$) this integral is zero. But also a rotating observer
of a motionless fluid should not see a rise in the temperature of a fluid
\footnote{We do not consider here the a rigidly rotating fluid as it os often done
in the literature, because a rigidly rotating fluid is unphysical.}
that means
\begin{align}
 \int_V \sigma'_{ij} \pd{r_j} v_i dV = 0 \label{eq:nodiss}
\end{align}
A rotating observer of a motionless fluid sees a velocity field of the form
\begin{align}
v_i=\epsilon_{ijk}\omega_j r_k
\end{align}
where $\omega_j$ is the angular velocity vector and $r_k$ is the position vector.
It can be shown, that for such a velocity field the jacobian is antisymmetric, that 
means
\begin{align}
\ppd{r_j}{v_i} = -\ppd{r_i}{v_j}
\end{align}
Using this and equation \eqref{eq:uascontr} in equation \eqref{eq:nodiss} we get
\begin{align}
\int_V \frac{1}{2}\lra{\sigma'_{ij}-\sigma'_{ji}} \pd{r_j} v_i dV = 0
\end{align}
This relation can only be fulfilled if the stress tensor $\sigma'_{ij}$
is symmetric
\begin{align}
\sigma'_{ij} = \sigma'_{ji}
\end{align}
For a newtonian fluid is it assumed that the stress tensor is proportional only to the
first derivatives of the velocity field. Together with the requirement of symmetry
the most general for of such a tensor is
\begin{align}
\sigma'_{ij} = a\lra{\ppd{r_i}{v_j}+\ppd{r_j}{v_i}}+b \delta_{ij}\ppd{r_k}{v_k}
\end{align}
Usually the trace is split off the first term and added the second term so
\begin{align}
\sigma'_{ij} = a \lra{\ppd{r_i}{v_j}+\ppd{r_j}{v_i}
-\frac{2}{3}\delta_{ij}\ppd{r_k}{v_k}}+\lra{\frac{2 a}{3}+b}\delta_{ij}\ppd{r_k}{v_k}
\end{align}
Using the definitions $2a = \eta$ and $\frac{2 a}{3}+b=\zeta$ we get the form
most common in literature
\begin{align}
\sigma'_{ij} =  
\eta\lrb{\frac{1}{2}\lra{\ppd{r_j}{v_i}+\ppd{r_i}{v_j}}
-\frac{1}{3}\delta_{ij}\ppd{r_k}{v_k}}
+\zeta \delta_{ij}\ppd{r_k}{v_k}
\end{align}

\section{Stress tensor in cartesian coordinates for 1d, 2d and 3d}
\label{stress1d2d3d}
For a so called newtonian fluid it can be shown, that the stress tensor
$\sigma'_{ij}$ in cartesian coordinates in $n$ dimensions is of the form
\begin{align}
\sigma'_{ij}=2\eta S^*_{ij}+\zeta\delta_{ij}\pd{r_k}v_k 
\end{align}
with $S^*_{ij}$ being the symmetric tracefree part of the tensor $\pd{x_j}v_i$
\begin{align}
S^*_{ij}=S_{ij}-\frac{1}{n}\delta_{ij}\pd{r_k}v_k
\end{align}
and $S_{ij}$ being the so called rate-of-strain tensor
\begin{align}
\frac{1}{2}\lra{\pd{r_j}v_i+\pd{r_i}v_j}
\end{align}
In 1d cartesian coordinates the component of the 
symmetric tracefree part of the stress tensor $S^*_{ij}$ is  
\begin{align}
S_{xx}=\frac{1}{2}\lra{\pd{r_x}v_x+\pd{r_x}v_x}-\frac{1}{1}\pd{r_x}v_x = 0.
\end{align}
In 2d cartesian coordinates the components of $S^*_{ij}$ are
\begin{align}
S_{xx}&=\frac{1}{2}\lra{\pd{r_x}v_x+\pd{r_x}v_x}
		 -\frac{1}{2}\lra{\pd{r_x}v_x+\pd{r_y}v_y}\\
		&=\frac{1}{2}\lra{\pd{r_x}v_x-\pd{r_y}v_y}, \\
S_{xy}&=\frac{1}{2}\lra{\pd{r_y}v_x+\pd{r_x}v_y}, \\
S_{yx}&=S_{xy},\\
S_{yy}&=\frac{1}{2}\lra{\pd{r_y}v_y-\pd{r_x}v_x}.
\end{align}
In 3d cartesian coordinates the components of $S^*_{ij}$ are
\begin{align}
S_{xx}&=\frac{1}{2}\lra{\pd{r_x}v_x+\pd{r_x}v_x}
		 -\frac{1}{3}\lra{\pd{r_x}v_x+\pd{r_y}v_y+\pd{r_z}v_z}\\
		&=\frac{1}{3}\lra{2\pd{r_x}v_x-\pd{r_y}v_y-\pd{r_z}v_z},\\
S_{xy}&=\frac{1}{2}\lra{\pd{r_y}v_x+\pd{r_x}v_y},\\
S_{xz}&=\frac{1}{2}\lra{\pd{r_z}v_x+\pd{r_x}v_z},\\
S_{yx}&=S_{xy},\\
S_{yy}&=\frac{1}{3}\lra{2\pd{r_y}v_y-\pd{r_x}v_x-\pd{r_z}v_z},\\
S_{yz}&=\frac{1}{2}\lra{\pd{r_z}v_y+\pd{r_y}v_z},\\
S_{zx}&=S_{xz},\\
S_{zy}&=S_{yz},\\
S_{zz}&=\frac{1}{3}\lra{2\pd{r_z}v_z-\pd{r_x}v_x-\pd{r_y}v_y}.\\
\end{align}

\section{Stress tensor in comoving coordinates}\label{costress}
With the help of transformation (\ref{eq:cotrans4}) and (\ref{eq:cotrans5}) we
can transform the stress tensor for a newtonian fluid in cartesian coordinates
\begin{align}
\sigma'_{ij}= 2\eta\lrb{\frac{1}{2}\lra{\pd{r_j}v_i+\pd{r_i}v_j}
-\frac{1}{n}\delta_{ij}\pd{r_k}v_k}+\zeta\delta_{ij}\pd{r_k}v_k.
\end{align}
into the stress tensor for a newtonian fluid in comoving coordinates
\begin{align}
\sigma'_{ij}=& 
2\eta\lrb{\frac{1}{2} 
\lra{\fa\pd{x_j}u_i+\fh\delta_{ij}+\fa\pd{x_i}u_j+\fh\delta_{ji}} 
-\frac{1}{n}\delta_{ij}\lra{\fa\pd{x_k}u_k+n\fh}}\\
&+\zeta\delta_{ij}\lra{\fa\pd{x_k}u_k+n\fh}\\
=&2\eta\lrb{\frac{1}{2 a} 
\lra{\pd{x_j}u_i+\pd{x_i}u_j}+\fh\delta_{ij}-\frac{n}{n}\fh\delta_{ij}
-\frac{1}{n a}\delta_{ij}\pd{x_k}u_k}\\
&+\zeta\delta_{ij}\lra{\fa\pd{x_k}u_k+n\fh}\\
=&\fa\lrc{2\eta\lrb{\frac{1}{2} 
\lra{\pd{x_j}u_i+\pd{x_i}u_j}-\frac{1}{n}\delta_{ij}\pd{x_k}u_k}
+\zeta\delta_{ij}\lra{\pd{x_k}u_k+n\dot{a}}}.
\end{align}

\section{A contradiction when computing the dissipation}\label{contradiss}
The dissipation for an newtonian incompressible fluid is
\begin{align}
\mathcal{E}
=-\frac{1}{V} \int_V \frac{\eta}{2}\lra{\ppd{r_j}{v_i}+\ppd{r_i}{v_j}}^2 dV
\end{align}
If we assume a flowfield with $\ppd{r_x}{v_x} \neq 0$ (which causes
$\ppd{r_y}{v_y}+\ppd{r_z}{v_z}=-\ppd{r_x}{v_x}$, because the divergence
of the velocity field must be zero) and vanishing diagonal components of
the jacobian 
$\ppd{r_x}{v_y}=\ppd{r_y}{v_x}=\ppd{r_y}{v_z}=\ppd{r_z}{v_y}=\ppd{r_z}{v_x}=\ppd{r_x}{v_z}=0$
, we compute via the formula above a
dissipation of
\begin{align}
\mathcal{E}
=-\frac{2 \eta}{V} \int_V \lra{\ppd{r_x}{v_x}}^2 +\lra{\ppd{r_y}{v_y}}^2 +
\lra{\ppd{r_z}{v_z}}^2 dV. 
\end{align}
The absolut value of the dissipation must be greater than zero, because 
all the terms in the integral are quadratic and our assumption was
$\ppd{r_x}{v_x},\ppd{r_y}{v_y},\ppd{r_z}{v_z}  \neq 0$. 

But if we compute the dissipation according to equation \eqref{eq:nicdiss}
like
\begin{align}
\mathcal{E} = -\frac{\eta}{V} \int_V  \omega^2 dV
\end{align}
we get for our flow field $\mathcal{E} = 0$, because the vorticity of
our velocity field is zero! This seems to be an obvious contradiction,
and either raises some doubts about the validity of the derivation of equation
\eqref{eq:nicdiss} for example in \citet{Frisch1995} or implies that our 
assumed flow field is unphysical.


\section{Structure functions}
A structure function of order $p$ is defined as\footnote{See
\citet{Pope2000}.}
\begin{align}
S_p(f(x))=\fil{\lrb{f(x+x')-f(x')}^p} = \ft \iinf \lrb{f(x+x')-f(x')}^p
dx'
\end{align}
The second order structure function is related to the spectrum
$\abs{F(k)}$ of the function $f$ like
\begin{align}
\ft \iinf \lrb{f(x+x')-f(x')}^2 dx' = \frac{2}{\sqrt{2\pi}}\iinf (1-e^{ikx})
\abs{F(k)}^2 dk \label{eq:structspec}
\end{align}
which can be proved\footnote{A sketch of this prove can also be found in
\citet[Appendix G]{Pope2000}.} by expanding the second order structure function 
\begin{align*}
&S_2(f(x))=\ft \iinf \lrb{f(x+x')-f(x')}^2 dx'\\
&= \ft \lrb{\iinf \abs{f(x+x')}^2 dx' 
- 2 \iinf f^*(x')f(x+x')dx'
+ \iinf \abs{f(x')}^2 dx'}.
\end{align*}
Substituting $x''=x+x'$ in the first term we get
\begin{align*}
S_2(f(x)) &= \ft \lrb{\iinf \abs{f(x'')}^2 dx'' 
- 2\iinf f^*(x')f(x+x')dx'
+\iinf \abs{f(x')}^2 dx'}\\
&= \frac{2}{\sqrt{2\pi}} \lrb{\iinf \abs{f(x')}^2 dx'-\iinf f^*(x')f(x+x')dx'}.
\end{align*}
Using Parseval's and the Wiener-Khichnin theorem we obtain the final result
\begin{align*}
&S_2(f(x))=\frac{2}{\sqrt{2\pi}} \lrb{\iinf \abs{F(k)}^2 dk
-\iinf \abs{F(k)}^2 e^{ikx} dk} \\
&= \frac{2}{\sqrt{2\pi}}\iinf (1-e^{ikx})\abs{F(k)}^2 dk. 
\end{align*}

The structure functions used in the theory of Kolmogorov
are the so called longitudinal structure functions of the velocity, which are
defined like
\begin{align}
S_2(v_{\parallel}(l)) =  
\fil{\lra{\lrb{\vec{v}(\vec{x}+\vec{l})-\vec{v}(\vec{x})}\cdot
\frac{\vec{l}}{l}}^p} =
\fil{\lra{v_{\parallel}(\vec{x}+\vec{l})-v_{\parallel}(\vec{x})}^p}
\end{align}
They are related to the longitudinal velocity
spectrum\footnote{In the literature this is often called
kinetic energy spectrum, but this is only true for
inkompressible flows.} $\abs{V_{\parallel}(k)}^2$ via equation
\eqref{eq:structspec}.
Sometimes also second order transvers structure functions are measured.
These are defined as
\begin{align}
S_2(v_{\perp}(l)) =  
\fil{\lra{\frac{\abs{\lrb{\vec{v}(\vec{x}+\vec{l})-\vec{v}(\vec{x})}\times
\vec{l}}}{l}}^p}.
\end{align}
The behaviour of the second order transvers structure functions for
homogeneous turbulence is uniquely determined by the longitudinal structure
function \citep[p. 192, Eqs. (6.28)]{Pope2000}. They also show the
characteristic $2/3$-slope as predicted for the longitudinal structure functions
\citep[p.60]{Frisch1995}.

In general structure functions of vectorial quantities like the velocity are
tensors, e.g. the general second order structure function of the velocity can
be defined as
\begin{align}
S_{ij}(\vec{x},\vec{l}) =  
\fil{\lrb{v_i(\vec{x}+\vec{l})-v_i(\vec{x})}
\lrb{v_j(\vec{x}+\vec{l})-v_j(\vec{x})}}
\end{align}
But it can be shown, that for local isotropy only the longitudinal
structure function $S_2(v_{\perp}(l))=S_{11}$ and the transversal structure
$S_2(v_{\perp}(l))=S_{22}=S_{33}$ are unequal zero \citep{Pope2000}. Since 
the transvers structure function is determined by the longitudinal structure
function in case of local homogeneity, for homogeneous and isotropic 
turbulence $S_{ij}$ is determined by the single scalar
function $S_{11} = S_2(v_{\parallel}(l))$ \citep{Pope2000}.

The third order structure function used in Kolmogorov theory is
defined as 
\begin{align}
S_{111}(\vec{x},\vec{l}) = \fil{\lrb{v_1(\vec{x}+\vec{l})-v_1(\vec{x})}^3}
\end{align}
which is often simply called $S_3(v(l))$. So the famous four-fifths law of
Kolmogorov is actually true only for one component of the third order
structure function tensor, but again for homogeneous and isotropic turbulence
the third order structure function tensor $S_{ijk}$ is uniquely determined by
the single scalar function $S_{111}=S_3(v(l))$.


\section{Some speculations about ...}
\subsection{Entropy}\label{entro}
The second law of thermodynamics states that the total entropy of any isolated
thermodynamic system tends to increase over time, approaching a maximum value
(not an infinite value!). But the higher the entropy, the lower the free,
useable energy of a system. Therefore it is perhabs easier to formulate the
second law of thermodynamics the other way round \citep{Feynman1967}: the free,
useable energy of any isolated thermodynamic system tends to decrease over time,
approaching a minimum value (zero?!). In this sense we can also understand 
Penrose \citep{Penrose1989}. He pointed out, that a closed self-gravitating
system will collapse to a black hole, which is the state with the maximum
entropy of the system. But this state is not a very unordered state
as one often imagines states with high entropy. But it is the state where no
free energy is left, the potential energy of the system is zero and so no
directed kinetic energy can be produced any more. Thats why the "ordered state"
of a black hole has the biggest entropy.

\subsection{Newtonian gravity}
Newtonan gravity is often expressed in form of the poisson equation
\begin{align}
\Delta \phi = 4\pi G\rho
\end{align}
Nevertheless this is equivalent to
\footnote{K is a constant with units $\unit{kg\ s\ m^{-2}}$. The role of
$K$ is similar to the role of $\epsilon$ in the electromagnetic theory.}
\begin{align}
\nabla \cdot \vec{g} = - 4\pi G\rho \label{eq:maxgrav1} \\
\nabla \times (K\vec{g}) = \vec{0},\label{eq:maxgrav2}
\end{align}
because $\nabla \times (-\nabla \phi) = \vec{0}$ and therefore we are able to
express the gravitational field as a gradient of a potential like
$\vec{g}=-\nabla \phi$. As we can see these equations are similar to Maxwells
equations for the electric field in the electrostatic case
($\ppd{t}{\vec{B}}=0$)
\begin{align}
\nabla \cdot (\epsilon \vec{E}) = - \frac{1}{\epsilon_0} \rho \\
\nabla \times \vec{E} = \vec{0}
\end{align}
and therefore we might call the equations \eqref{eq:maxgrav1} and
\eqref{eq:maxgrav2} Maxwells equations of gravity.

The following is just speculation:

One cannot derive the continuity equation from these equations
\begin{align*}
\ppd{t}{\rho} + \nabla \cdot \lra{\vec{v} \rho} &= 0 \\
\Leftrightarrow 
-\frac{1}{4\pi G}\pd{t}\lra{\nabla\cdot\vec{g}} 
+ \nabla \cdot \lra{\vec{v} \rho} &= 0 \\
\Leftrightarrow
-\frac{1}{4\pi G}\nabla\cdot\dot{\vec{g}}
+\nabla \cdot \lra{\vec{v} \rho} &= 0 \\
\Leftrightarrow 
\nabla \cdot \lra{\vec{v} \rho - \frac{1}{4\pi G} \dot{\vec{g}}} &= 0
\end{align*}
But if we fixed equation \eqref{eq:maxgrav2} in analogy to Maxwell by writing
\footnote{The units of the left hand side and the right hand side are the
same, because of our choice of units for $K$.}
\begin{align}
\nabla \times (K \vec{g}) = \vec{v} \rho - \frac{1}{4\pi G} \dot{\vec{g}},
\end{align}
we could derive the continuity equation from our systems of equations like
\begin{align*}
\nabla \cdot \lra{\nabla \times (K \vec{g})} &= \nabla \cdot \lra{\vec{v} \rho 
- \frac{1}{4\pi G} \dot{\vec{g}}}\\
\Leftrightarrow 
0 &= \nabla \cdot \lra{\vec{v} \rho} 
+ \pd{t}\lra{-\frac{1}{4\pi G} \nabla \cdot \vec{g}} \\
\Leftrightarrow
0 &= \nabla \cdot \lra{\vec{v} \rho} + \pd{t} \rho
\end{align*}
So instead of \eqref{eq:maxgrav1} and \eqref{eq:maxgrav2} we could have
\begin{align}
\nabla \cdot \vec{g} = - 4\pi G\rho, \\
\nabla \times (K\vec{g}) = \vec{v} \rho - \frac{1}{4\pi G} \dot{\vec{g}}.
\end{align}
In vacuum ($\rho=0$) we would have 
\begin{align}
\nabla \cdot \vec{g} = 0, \\
\nabla \times (K\vec{g}) = - \frac{1}{4\pi G} \dot{\vec{g}}.
\end{align}
From these vacuum equations we can try to derive a wave equation in analogy 
to the wave equation for the electromagnetic field like
\begin{align}
\nabla \times \lra{\nabla \times (K\vec{g})} = 
- \frac{1}{4\pi G} \lra{\nabla \times \dot{\vec{g}}}= 
- \frac{1}{4\pi G} \pd{t} \lra{\nabla \times \vec{g}}=
+ \frac{1}{(4\pi G)^2 K} \frac{\partial^2}{\partial t^2} \vec{g}
\end{align}
and
\begin{align}
\nabla \times \lra{\nabla \times (K \vec{g})} = 
K \nabla \lra{\nabla \cdot \vec{g}} - K \Delta \vec{g}
\end{align}
With $\nabla \cdot \vec{g} = 0$ it follows that
\begin{align}
\frac{\partial^2}{\partial t^2} \vec{g}+(4\pi G K)^2 \Delta \vec{g} = 0
\end{align}
This could be interpreted as a wave equation with complex velocity
$c=i \cdot 4\pi G K$.

\subsection{The divergence equation}\label{diveq}
\subsubsection{General fluid}
We start with the momentum equation \eqref{eq:mom}
\begin{align*}
\pd{t}(\rho v_i) + \pd{r_j}(v_j \rho v_i) &= -\pd{r_i}p + \pd{r_j}\sigma'_{ij}
-\rho \pd{r_i} \phi
\end{align*}
where we assumed $g_i=-\pd{r_i} \phi$. If we make the substitutions 
$\pd{r_i} p \rightarrow \pd{r_j} p \delta_{ij}$ and 
$\pd{r_i} \phi \rightarrow \pd{r_j} \phi \delta_{ij}$ we can write it in the 
form
\begin{align*}
\pd{t}(\rho v_i) + \pd{r_j}(v_j \rho v_i + p \delta_{ij} - \sigma'_{ij}) 
= -\rho \pd{r_j} \phi \delta_{ij}
\end{align*}
Taking the divergence of this equation we get
\begin{align*}
\pd{t}\lrb{\pd{r_i} (\rho v_i)} 
+ \frac{\partial^2}{\partial r_i \partial r_j}(v_j \rho v_i + p \delta_{ij} - \sigma'_{ij}) 
= -\pd{r_i}\lra{\rho \pd{r_j} \phi \delta_{ij}}
\end{align*}
where we assumed that $\pd{t}$ and $\pd{r_i}$ commute. Using the continuity 
equation \ref{eq:mass} we get a interesting form of the fluiddynamic 
equations
\begin{align}
\pdd{t}\rho 
- \frac{\partial^2}{\partial r_i \partial r_j}(v_j \rho v_i + p \delta_{ij} - \sigma'_{ij}) 
= +\pd{r_i}\lra{\rho \pd{r_j} \phi \delta_{ij}}\label{eq:divbeauty}
\end{align}
In case of no gravitation, the fluiddynamic equation can be written in a form
showing some similarity to a wave equation
\begin{align}
\pdd{t}\rho 
- \frac{\partial^2}{\partial r_i \partial r_j}(v_j \rho v_i + p \delta_{ij} - \sigma'_{ij}) 
= 0
\end{align}
But despite of its simple form, this equation hides an extreme complexity.

Neglecting the stress tensor $\sigma'_{ij}$ and solving for pressure this
equation is written like
\begin{align}
\pdd{r_i} p = \pdd{t}\rho- 
\frac{\partial^2}{\partial r_i \partial r_j}(\rho v_i v_j)\label{eq:instpres}
\end{align}
and sometimes called equation for the instantaneous pressure.

 
\subsubsection{Incompressible newtonian fluid}
If we have a newtonian fluid 
($\sigma'_{ij}=\eta \lra{\ppd{r_j}{v_i}+\ppd{r_i}{v_j}}$) with $\rho=const$ in 
time and space (which leads to $\ppd{r_i}{v_i}=0$ and also
$\frac{\partial^2}{\partial r_i \partial r_j}\sigma'_{ij}=0$) 
equation \eqref{eq:divbeauty} simplifies to
\begin{align}
\frac{\partial^2}{\partial r_i \partial r_j}\phi = 
- \frac{1}{\rho} \frac{\partial^2}{\partial r_i \partial r_j} \lra{p \delta_{ij} + \rho v_i v_j}
\end{align}
which can be intepreted as an equation for the gravitational potential of a 
fluid with constant(!) density:
\begin{align}
\Delta \phi = - \frac{1}{\rho} \frac{\partial^2}{\partial r_i \partial r_j} 
\lra{p \delta_{ij} + \rho v_i v_j}
\end{align}
So this equation seems to imply, that pressure and velocity stresses can be a 
source of gravity not only in general relativity, but also in a newtonian 
framework. We can exploit the relation to general relativity further by 
introducing the Stress-Energy-Tensor $T_{ij} = p \delta_{ij} + \rho v_i v_j $ 
\begin{align}
\Delta \phi = - \frac{1}{\rho} \frac{\partial^2}{\partial r_i \partial r_j} T_{ij} \label{eq:gravdiv}
\end{align}
This looks similar to the Einstein equation
\begin{align}
G_{\mu \nu} = \frac{8\pi G}{c^4} T_{\mu \nu}
\end{align}
except that we have higher derivatives of the stress energy tensor on the 
right hand side. 

If we substitute again $\phi \rightarrow \phi \delta_{ij}$ we can write equation 
\eqref{eq:gravdiv} like
\begin{align}
\frac{\partial^2}{\partial r_i \partial r_j} 
\lra{\frac{p}{\rho} \delta_{ij} + v_i v_j +  \phi \delta_{ij}}=0
\end{align}
which looks like a tensor version of Bernoullis law:
\begin{align}
\pd{r_i}\lra{\frac{p}{\rho} + \frac{1}{2} v^2 +  \phi} = 0
\end{align}

\subsection{The limit $\nu \longrightarrow 0$}
At first we should mention that the limit $\nu \longrightarrow 0$ is not
equivalent to the limit $Re \longrightarrow \infty$. The definition of the
Reynoldsnumber is\footnote{Just from looking at the units we could also write
$Re=\frac{\text{Area per time}}{nu}$ or 
$Re=\frac{\rho L V}{\eta} = \frac{\rho V^2 t}{\eta} 
= \frac{\text{Action density}}{\eta}$.} 
\begin{align}
Re = \frac{L V}{\nu}
\end{align}
where $L$ is a characteristic length and $V$ is a characteristic velocity 
of the system (whatever that means). Therefore the limit 
$Re \longrightarrow \infty$ can mean 
$\nu \longrightarrow 0$, but also $L \longrightarrow \infty$ or 
$V \longrightarrow \infty $. So if we would make an experiment with the same
fluid and the same setup, but would only increase the (characteristic) speed
of the fluid, we would measure effects for higher and higher Reynoldsnumbers.
But this has nothing to do with the limit $\nu \longrightarrow 0$, because 
we not only have a higher Reynoldsnumber, but also have a higher Machnumber.
This would mean, that we would also measure more and more effects due to the
the increasing compressibility of our fluid. Therefore when doing an
experiment, that should give insights into the regime of vanishing viscosity
we cannot simply increase the Reynoldsnumber by increasing the 
characteristic velocity. We  have to take care that all the other
characteristic numbers (Mach-Number, Froude-Number ...) stay the same. 
Otherwise we measure effects that have nothing to with the interesting limit
$\nu \longrightarrow 0$. Sadly the limit $\nu \longrightarrow 0$ is often mixed
up with $Re \longrightarrow \infty$ in the literature.

For understanding the limit $\nu \longrightarrow 0$ \citet{Feynman1964}
investigates the vorticity equation for a newtonian incompressible fluid
\begin{align}
\pd{t}\omega_g
-\epsilon_{ghi}\pd{r_h} \epsilon_{ijm} v_j \omega_m =
\nu \pdd{r_j}\omega_g
\end{align}
First he mentions that for the case of very high viscosity 
\begin{align}
\nu \pdd{r_j}\omega_g \gg \pd{t}\omega_g
-\epsilon_{ghi}\pd{r_h} \epsilon_{ijm} v_j \omega_m.
\end{align}
Therefore the left side of the vorticity equation can be neglected and 
the problem describing a fluid with high viscosity can be simplified to solving
the so called Stokes equation\footnote{$0_g$ is a component of the zero vector.}
\begin{align}
\pdd{r_j}\omega_g =  0_g
\end{align}
But what happens for very low viscosity? \citet{Feynman1964} says decreasing
the viscosity of a fluid leads to an increase of the velocity fluctuations and
so the increasing factor $\pdd{r_j}\omega_g$ compensates the smallness of
the viscosity. The product of viscosity and $\pdd{r_j}\omega_g$ doesn't go to
the limit $0_g$, which we would expect from the equation of vorticity we get
for an ideal fluid
\begin{align}
\pd{t}\omega_g
-\epsilon_{ghi}\pd{r_h} \epsilon_{ijm} v_j \omega_m =0_g  
\end{align}
So the equations for an ideal fluid do not(!) yield the right limit for
vanishing viscosity. Can we find other equations that do give the right limit
for vanishing viscosity? 

One idea might be the following: \citet{Feynman1964} said that the limit
of $\nu \pdd{r_j}\omega_g$ is not $0_g$, but what is it then? The
easiest alternative would be a constant vector $C_g \neq 0$! This leads us to
the equations
\begin{align}
\pd{t}\omega_g
-\epsilon_{ghi}\pd{r_h} \epsilon_{ijm} v_j \omega_m = C_g 
\label{eq:lowvisvort} 
\end{align}
and(!)
\begin{align}
 \pdd{r_j}\omega_g = \frac{1}{\nu} C_g
\label{eq:lowvisvort2} 
\end{align}
If we ignore the second term on the left hand side of equation
\eqref{eq:lowvisvort} for a moment, we see that $C_g$ stand for
a dissipation of vorticity independent of $\nu$. Something we might expect
for a fluid with low viscosity. The second equation \eqref{eq:lowvisvort2} is
more confusing, since it is a second order partial differential equation,
dependent on viscosity and shows no dependency on time
\footnote{If $\vec{C}=\vec{C}(x,t)$, then it would also depend on time}.
Maybe we can intepret the existence of the two equations in the sense that 
equation \eqref{eq:lowvisvort} gives the vorticity for $\nu=0$ and therefore
must be independent of $\nu$ and equation \eqref{eq:lowvisvort2} gives the
vorticity for a very small but not zero viscosity and therefore is still
dependent on the viscosity.

Nevertheless it is interesting that we can solve equation
\eqref{eq:lowvisvort2} if we know $\vec{C}(\vec{x})$, because it is a 
vector-poisson equation well know from electrodynamics.
Written in vector notation equation \eqref{eq:lowvisvort2} is
\begin{align}
\Delta \vec{\omega} = \frac{1}{\nu} \vec{C}(\vec{x})
\end{align}
The solution to this equation is
\begin{align}
\vec{\omega} = 
\frac{1}{4 \pi \nu} 
\int \frac{\vec{C}(\vec{x}')}{\abs{\vec{x}-\vec{x}'}} dV'
\end{align}
From this we can compute the velocity field because we know
\begin{align}
\nabla \cdot \vec{v} = 0,&& \nabla \times \vec{v} = \vec{\omega}
\end{align}
So the velocity is a so called pure curl field, because the divergence
of the velocity is zero. Following Bronstein, p. 665 we make the ansatz
\begin{align}
\vec{v}= \nabla \times \vec{A},&& \nabla \cdot \vec{A} = 0
\end{align}
That means
\begin{align}
\nabla \times (\nabla \times \vec{A}) &= \vec{\omega}\\
\Leftrightarrow \nabla(\nabla \cdot \vec{A}) - \Delta \vec{A} &= \vec{\omega}\\
\Rightarrow \Delta \vec{A} &= -\vec{\omega}
\end{align}
So again this leads to a vector-poisson equation for our vectorfield $\vec{A}$.
The complete solution for equation \eqref{eq:lowvisvort2} is then
\begin{align}
\vec{v}= \nabla \times \vec{A} 
\end{align}
with
\begin{align}
\vec{A} &=\frac{1}{4 \pi} 
\int \frac{\vec{\omega}(\vec{x}')}{\abs{\vec{x}-\vec{x}'}} dV'\\
&=\frac{1}{4 \pi} 
\int \frac{-\frac{1}{4 \pi \nu} 
\int \frac{\vec{C}(\vec{x}'')}{\abs{\vec{x'}-\vec{x}''}} dV''}
{\abs{\vec{x}-\vec{x}'}} dV'\\
&= -\frac{1}{16 \pi^2 \nu} \int \frac{1}{\abs{\vec{x}-\vec{x}'}}
\int \frac{\vec{C}(\vec{x}'')}{\abs{\vec{x'}-\vec{x}''}} dV'' dV'
\end{align}

\bibliographystyle{apalike}
\bibliography{promotion}

\end{document}



\appendix
%\input{appendix/appendix}

% try plainnat, abbrvnat, unsrtnat, apalike, aa, astroads

\bibliographystyle{apalike}
\bibliography{promotion}

\end{document}

